\documentclass[12pt]{article}
\usepackage{xspace}
\usepackage{listings}
\usepackage{color}

\title{RJava User and Developer Manual}

\author{Yi Lin\\yi.lin@anu.edu.au}

\begin{document}

% macros
\newcommand{\rjc}{RJC\xspace}
\newcommand{\rjcfull}{RJava Compiler\xspace}

% java code format
% use \begin{lstlisting} your code \end{lstlisting}
\definecolor{dkgreen}{rgb}{0,0.6,0}
\definecolor{gray}{rgb}{0.5,0.5,0.5}
\definecolor{mauve}{rgb}{0.58,0,0.82}

\lstset{frame=tb,
  language=Java,
  aboveskip=3mm,
  belowskip=3mm,
  showstringspaces=false,
  columns=flexible,
  basicstyle={\small\ttfamily},
  numbers=none,
  numberstyle=\tiny\color{gray},
  keywordstyle=\color{blue},
  commentstyle=\color{dkgreen},
  stringstyle=\color{mauve},
  breaklines=true,
  breakatwhitespace=true
  tabsize=3
}

\newcommand{\javacode}[1]{{\lstinline@#1@}}

% cmd
\definecolor{MyGray}{rgb}{0.78,0.78,0.78}
\makeatletter\newenvironment{cmd}{%
   \newline
   \sffamily
   \noindent
%   \begin{quotation}
   \begin{lrbox}{\@tempboxa}
   \begin{minipage}{0.8\columnwidth}}{\end{minipage}\end{lrbox}%
   \colorbox{MyGray}{\usebox{\@tempboxa}}
   \newline
%   \end{quotation}
}\makeatother

\maketitle

\begin{abstract}
% rjava
RJava is a restricted subset of the Java language 
with low-level extensions
that allow access to hardware and operating system. 
% rjava benefits
RJava utilizes the same syntax as Java, and consequently
inherits benefits from Java such as type safety, 
various software engineering tools and productivity. Futhermore, 
by restrictions, RJava is a fully static language with closed world
assumption. Thus it requires a much more succinct runtime, and
is well suitable for aggressive static compilation and optimizations. 
% rjava use
RJava is designed to be an implementation language for virtual machine
construction (and more broadly for system programming). 

This manual describes the language and its 
current implementation--the \rjcfull~(\rjc). It is intended for RJava users
and developers who are willing to contribute. This manual will be
maintained to keep pace with the \rjc code base. 
\end{abstract}
\clearpage

\setcounter{secnumdepth}{5}
\setcounter{tocdepth}{5}
\tableofcontents 
\clearpage

%%%%%%%%%%%%%%%%%%
% RJava user manual     %
%%%%%%%%%%%%%%%%%%

% 1. RJava Basics
\section{RJava Basics}

This section describes about RJava's restrictions and extensions, 
which make RJava differentiate from the vanilla Java. 

\paragraph*{What is RJava?}
\begin{quotation}
Generally speaking, unless what is \emph{restricted} and 
what is \emph{extended},
the other part of RJava remains the same as Java (
syntactically and semantically) . Besides,
the restrictions and the extensions are also
compliant with Java \emph{syntax}. 
\end{quotation}

\noindent
Thus, Java compilers and static analyzers can 
parse RJava code (they are syntactically same).
Java runtimes can
execute RJava programs as well; However, Java runtimes
will not be able to recognize RJava
specific restrictions and extension, thus executing
RJava with Java runtimes may result in wrong results, which
is not encouraged. 

RJava is also compliant with existing Java editors and IDEs. 
When using Java IDE to edit RJava code, extensions and restrictions
need to be imported and also try avoid using execution from IDE. 

\subsection{RJava Restrictions}

One important feature about RJava is that all its restrictions 
are formalized and any RJava code need to declare the restrictions
it complies with. The benefit is that both the developers
and the RJava compilers will be able to see the declaration
and ensure the restrictions are met. 

\subsubsection{Restriction Rules and Rulesets}

RJava restrictions are defined as Java annotations, and they are
annotated by @\javacode{RestrictionRule (org.rjava.restriction.RestrictionRule)}. 
The following is how @\javacode{NoException} restriction is defined:

\begin{lstlisting}
@RestrictionRule
public @interface NoException{
}
\end{lstlisting}

Restriction rules can be used to annotate classes or methods, in the same way as
Java annotations are used. However, it is \emph{not} encouraged to directly use
restriction rules. The preferred way is to form a @\javacode{RestrictionRuleset} by
a set of chosen @\javacode{RestrictionRule}, and annotate a certain scope of code 
with such @\javacode{RestrictionRuleset}. 

One example is the MMTk scope and its restriction @\javacode{MMTk}. The following
ruleset indicates that any code within @\javacode{MMTk} needs to obey 
@\javacode{RJavaCore} ruleset (any RJava code need to include this ruleset, it will be
described in next subsection) as well as two additional restrictions
(@\javacode{NoRuntimeAllocation} and @\javacode{Uninterruptible}). The
declaration is as below:

\begin{lstlisting}
@RestrictionRuleset

@RJavaCore
@NoRuntimeAllocation
@Uninterruptible
public @interface MMTk{
}
\end{lstlisting}

\subsubsection{Restriction Model}
Restrictions follow those rules:

\begin{itemize}

\item 
Restriction rules and restriction rulesets apply to scopes of
classes and methods to declare restrictions, \emph{not} to fields. 
\item
Restriction rules and rulesets may apply to another restriction ruleset,
to indicate that the latter ruleset \emph{includes} those rules and rulesets.
This may happen iteratively. 
\item
Any RJava class needs to be restricted by @\javacode{RJavaCore}, or restricted
by a ruleset that (iteratively) includes @\javacode{RJavaCore}.
\item
A scope restricted by a restriction ruleset is restricted by every single
restriction rule within that ruleset. 
\item
If a class is restricted by Rule A, all its methods are restricted by Rule A 
unless the method is restricted otherwise 
(particularly @\javacode{Uninterruptible} vs. @\javacode{Interruptible}). 
\item
Restrictions on a class do \emph{not} affect its child classes. Restrictions on a method
do \emph{not} affect its overriding methods. 
\item
Restrictions on a class \emph{affect} its nested classes. As a result, 
those restrictions also \emph{affect} the methods of the nested class. 

\end{itemize}

\subsubsection{RJavaCore Ruleset}

@\javacode{RJavaCore} declares the basic restrictions that any RJava-compliant code
need to follow. It declares as:

\begin{lstlisting}
@RestrictionRuleset

@NoDynamicLoading
@NoReflection
@NoException
@NoCastOnMagicType
@NoExplicitLibrary
@NoEnum
public @interface RJavaCore {
}
\end{lstlisting}

\noindent
Specifically, 
\begin{itemize}

\item
@\javacode{NoDynamicLoading}: 
the code is not allowed to dynamically load classes. To forbid this, 
\javacode{java.lang.Class.forName()} and 
\javacode{java.lang.Classloader.loadClass()}
are not allowed. Also inheriting from {java.lang.Classloader} is also not allowed. 

\item
@\javacode{NoReflection}: 
the code is not allowed to use any reflection-based feature. To forbid this,
any method that may return \javacode{java.lang.Class} is not allowed. 
@\javacode{NoExplicitLibrary} already forbids the use of 
\javacode{java.lang.reflect} package. Futhermore, 
\javacode{java.lang.Object.getClass()} 
is not allowed (\javacode{.class} syntax is also not allowed, since 
on bytecode level, it gets translated into \javacode{Object.getClass()}. )

\item
@\javacode{NoException}:
the code is not allowed to throw exceptions. To forbit this, 
\javacode{throws} in method declarations and also
\javacode{throw} statement are not allowed. Catching blocks
are allowed in order to maintain correct syntax when using
library methods, however, there will be \emph{no} exception thrown and 
catching blocks will \emph{not}
get executed in any case. It is preferred to write such code
as below:
\begin{lstlisting}
Object lock = new Object();
try {
  lock.wait();
} catch (InterruptedException ignore) {}
\end{lstlisting}

\item
@\javacode{NoCastOnMagicType}:
the code is not allowed to do any type casting when 
\emph{at least one} side is RJava magic types. 
RJava magic types reside in \javacode{org.vmmagic.unboxed}
package, including 5 unboxed magic types (\javacode{Address}, 
\javacode{Extent}, \javacode{ObjectReference}, 
\javacode{Offset}, \javacode{Word}) and their array counterparts. 
Allowed type casting for magic types can be done via
provided methods in those classes 
(more will be discussed in the section about extensions). 

\item
@\javacode{NoExplicitLibrary}:
the code is not allowed to import any java library. However,
classes in the \javacode{java.lang.*} package are implicitly 
imported to any Java code, and they are entangled
with Java syntax. RJava \emph{allows} the use of 
\javacode{java.lang.*}. Current \rjcfull has only implemented
a subset of the \javacode{java.lang.*} package, 
more complete implementation will be done in the future. 

\item
@\javacode{NoEnum}:
the code is not allowed to use enumerate type. Enumerate
type is forbidden since enabling full features of 
Java \javacode{enum} would highly involve with dynamic behaviors
and extensive use of library methods from 
\javacode{java.lang.Enum}. We designed RJava to be 
a \emph{fully static} language, thus we forbid the 
use of enumerate type. 

\end{itemize}

These core restrictions define the RJava language. Any valid
RJava code needs to declare their compliance with @\javacode{RJavaCore}. 

By these restrictions, RJava $1)$ allows closed world assumption, 
which helps aggressive static compilation, $2)$ supports little dynamic behaviors
and bares very succinct execution runtime, which makes RJava possible to 
run with limited hardware resources, and $3)$ is a simple yet still expressive
language. 

\subsubsection{Other Predefined Restrictions}
RJava defines the above \emph{restriction model}. 
RJava encourages its users to utilize this model and, if favorable, add 
their own restriction rules and rulesets to more precisely describe
restrictions on their own scope. However, RJava predefined a set
of restrictions. Some are already mentioned in the sections above. 
These predefined restrictions are subject to change, check 
\javacode{org.rjava.restriction.rules} package in the code base for
the latest information. 

\subsection{RJava Extensions}

In order to undertake system programming task, RJava introduces
extensions to allow efficient access to hardware and operating 
systems. RJava extensions include $1)$ the \javacode{org.vmmagic}
package (by Daniel et al.) that allows memory/address representation and operations,
and $2)$ the \javacode{org.rjava.osext} package that provides 
access to operating systems (including some system calls). 

\subsubsection{org.vmmagic}

The \javacode{org.vmmagic} package provides 5 unboxed magic types
to describe pointer-alike types and their operations. Though implementation
may vary, `unboxed' means those types are not normal RJava objects,
and they are more like primitive types though there are methods declared 
for each type. 
These magic types share the same length as the pointer length on the target machine:

\begin{itemize}
\item
\javacode{ObjectReference}: can be cast from and to an RJava object. 
\emph{No} arithmetic, load/store or comparison operations are supported. 
\javacode{ObjectReference} can be cast to \javacode{Address} for 
further operations (unsafe). 

\item
\javacode{Address}: used as a pointer-alike type. 
Arithmetic, load/store and comparison operations are supported. 

\item
\javacode{Extent}: used to describe size in bytes (unsigned positive value)
Arithmetic and comparison operations are supported. 

\item
\javacode{Offset}: used to describe offset in bytes (signed). 
Arithmetic and comparison oeprations are supported. 

\item
\javacode{Word}: used as a pointer-sized integer. 
Arithmetic, comparison and bit-wised operations are supported. 
Can be cast from and to the other 4 magic types. 
\end{itemize}

\noindent
The above unboxed magic types each have an array counterpart, namely
\javacode{ObjectReferenceArray}, \javacode{AddressArray}, 
\javacode{ExtentArray}, \javacode{OffsetArray},
\javacode{WordArray}. 
They provide \javacode{create()}, 
\javacode{get()/set()} and \javacode{length()} operations. 
Arrays of those magic types should \emph{only} be created by using these types.

Besides magic types, \javacode{org.vmmagic} provides `pragma'
for supplying information to the compiler. Useful 
`pragmas' include @\javacode{Inline}, @\javacode{NoInline}, 
@\javacode{NoBoundsCheck}, @\javacode{NoNullCheck}, etc. Some `pragmas' 
from \javacode{org.vmmagic} are now considered as an 
RJava @\javacode{RestrictionRule} such as @\javacode{Uninterruptible}, 
and some are very specific to Java 
and not appliable for RJava thus will be deleted. This part is still a draft, and
subject to change. 

\subsubsection{org.rjava.osext}

The \javacode{osext} extension allows access to operating systems. Currently
\javacode{osext} only includes a minimum set of methods for implementing
a memory manager (MMTk), and it \emph{will} expand during further development. 

\begin{itemize}
\item
\javacode{OSConcurrency}: concurrency/threading related methods. 
  \begin{itemize}
  \item \javacode{void mutexLock/Unlock(Object lock)}: 
  provides an alternative to using synchronization on an RJava object. 
  \item \javacode{void threadSuspend/Resume(Thread t)}:
  suspending an RJava thread (unsafe)
  \end{itemize}

\item
\javacode{OSMemory}: memory related methods. 
  \begin{itemize}
  \item \javacode{Address malloc(int size)}:
  allocates raw memory
  \item \javacode{Address mmap(Address start, Extent length, int protection, int flags, int fd, Offset offset)}:
  calls mmap system call
  \item \javacode{int mprotect(Address start, Extent length, int prot)}:
  calls mprotect system call
  \item \javacode{Address memset(Address start, int c, Extent length)}:
  calls memset system call
  \end{itemize}

\item
\javacode{OSNative}: other methods that calls to native code
  \begin{itemize}
  \item \javacode{int errno()}: a wrap of errno() in errno.h
  \item \javacode{String strErrno()}: a warp of strerror()
  \item \javacode{double random()}: generates a random double between 0 and 1
  \end{itemize}

\end{itemize}

\subsection{Relation between RJava and MMTk/JikesRVM}

The RJava project was motivated by trying improve the portability of Memory Management
ToolKit (MMTk). MMTk started with JikesRVM project, and serves as its memory manager. 
The same as the rest part of JikesRVM, MMTk is also written in a variant of Java 
(ad-hoc restrictions with \javacode{org.vmmagic} extensions). 
Though MMTk was designed to be a portable language-agnostic memory manager, 
its portability was never a success. 

One of the main reasons that constrain its portability is the portability of the language it
is written in. On one hand, such Java variant is specially tailored, and requires support from 
its hosting runtime. It is no longer a `write once run everywhere' Java program, and
it cannot execute on a stock Java VM. Whoever wants to host MMTk needs to find a way
to execute the Java variant that MMTk is written in. 
On the other hand, MMTk is written in such an Java variant, and 
there is an inneglectable performance cost to integrate MMTk(Java) with a hosting runtime 
written in C/C++ which is the most common case. Past experiences (VMKit, Rotor, GHC)
of porting MMTk took an approach of ahead-of-time (AOT) compiling MMTk to native codes, which overcame
these two obstacles. However, \emph{none} of their approaches is general or reusable, and each
of them took a great effort. Furthermore, in term of performance, 
\emph{none} of the experiences was reported as a success (meanwhile MMTk hosted by JikesRVM
achieves excellent performance). 

Thus the idea of formalizing the Java variant that MMTk is written in 
and providing an AOT compiler to lower such language for effective
integration with low-level C/C++ code becomes a promising solution
to MMTk portability issues. This is one part of the motivations. 

Another important motivation that evolves the RJava project is the clear benefits
from using a higher-level language for virtual machine implementation. 
Compared with traditional approach of using C/C++, higher-level languages
provide benefits in safety and productivity, both of which are valuable in 
VM construction. However, existing experiences of using high-level languages
for VM construction (including JikesRVM and MMTk) all formed their own
variant of such high-level languages, which not only results in inability of code
reusing but also practically introduces metacircularity issues. As we 
already found out, metacircularity should be orthogonal to using high-level languages;
however, in practice so far, whenever high-level languages are used in virtual machine 
construction, the chosen implementation language is always the target language of 
the VM. Thus this project not only wants to settle the reusability/portability issues
but also wants to create a general implementation language
that can deliver high-level language benefits without falling into metacircular traps. 

In short, RJava originates from the ad-hoc coding pattern used in MMTk, and absorbs
the \javacode{org.vmmagic} package which is also used in MMTk. However, it evolves
to be a more general implementation languages to deliver higher-level language benefits
as well as good performance. 

In next section, we wil describe the usage of the \rjcfull, current implementation of RJava. 

\clearpage

% 2. RJava Compiler Tools
\section{\rjcfull Tools}

The \rjcfull (\rjc) is an ahead-of-time compiler for RJava. It parses RJava programs, checks
restriction compliance and translates into the target C code. Then the C backends could further
compile the generated code into binary or other form of instructions (e.g. LLVM IR). 

\subsection{Building \rjc}

The \rjcfull is written in Java. An ant build file is provided for automatic building. 
\begin{enumerate}
\item Go to \rjc root directory (\$root). 
\item Use
\begin{cmd}
ant -f mybuild.xml
\end{cmd}
After the building succeeds, 
in \$root/build, the following files
can be found:
  \begin{itemize}
  \item components/: a copy of external Java archives that \rjc uses;
  \item rjava\textunderscore ext/ : a copy of RJava extensions source file;
  \item rjava\textunderscore rt/ : a copy of RJava runtime source file;
  \item rjc.jar: the executable \rjcfull archive. 
  \end{itemize}
\end{enumerate}

\subsection{RJava Helloworld}

In this subsection, we show an example of how to compiling
the simplest RJava hello world into executable. 

\begin{enumerate}
\item Type in the source code as below, save it as `HelloWorld.java'. 
For simplicity, we put the source code in the same directory as rjc.jar
(under \$root/build). 
Note that RJava uses the same file name extension as Java (.java). 

\begin{lstlisting}
// HelloWorld.java
import org.rjava.restriction.ruleset.RJavaCore;

@RJavaCore
public class HelloWorld {
  public static void main(String[] args) {
    System.out.println("Hello RJava");
  }
}
\end{lstlisting}

\item Compiling RJava into C. 
\begin{cmd}
java -cp rjc.jar:rjava\textunderscore rt/:. org.rjava.compiler.RJavaCompiler \\
-rjava\textunderscore ext rjava\textunderscore ext/ \\
-rjava\textunderscore rt rjava\textunderscore rt/ \\
-soot\textunderscore jdk components/soot/ \\
-dir . \\
HelloWorld.java
\end{cmd}
This will invoke \rjcfull (the arguments passed to \rjc will be discussed later, 
a script is also provided to simplify arguments needed). 
By default, the generated code locates in output directory of current directory. 
The RJava runtime sources will also be copied to the output directory, so they
can be compiled along with the generated sources. And
a GNU Makefile will be generated for compiling C into binary. 

\item Compiling C into binary.
\begin{cmd}
cd output; make
\end{cmd}
By default, the Makefile will use `gcc' as the C backend to compile the generated
C files. The output binary in this case will be `HelloWorld' (named after the class
where main method is found). 

\end{enumerate}

\subsection{Full Command Line Options}

The basic use of the \rjcfull is
\begin{cmd}
java -cp rjc.jar:rjava\textunderscore rt/:. org.rjava.compiler.RJavaCompiler\newline
[-option value] [source file(s)]...
\end{cmd}

\noindent
Here lists all the command line options of the \rjcfull. 

\subsubsection*{Environment setting options}
\emph{(Note: the`rjc' script will free you from setting these options manually. )}
\begin{itemize}
\item
\textbf{-rjava\textunderscore ext PATH [REQUIRED] \\}
set PATH as the directory for RJava extension source files. PATH should contain
org/rjava/osext, org/rjava/restriction and org/vmmagic. \rjc requires this option 
to locate those extensions during parsing RJava sources.
\item
\textbf{-rjava\textunderscore rt PATH [REQUIRED] \\}
set PATH as the directory for RJava runtime source files. PATH should contain
the implementation of \javacode{java.lang.*} and other runtime source files. 
\rjc will copy the contents in this directory to the output directory, allowing
the runtime to be compiled and linked with RJava programs. 
\item
\textbf{-soot\textunderscore jdk PATH [REQUIRED] \\}
set PATH as the directory for the JDK jars that Soot needs. PATH should contain
two JDK jars, jce.jar and rt.jar. \rjc uses Soot to parse RJava source files, 
and Soot requires those two jars to locate Java's library. 
\end{itemize}

\subsubsection*{Source file options}
\begin{itemize}
\item
\textbf{-dir PATH1:PATH2:... [REQUIRED] \\}
set PATH as the classpath of RJava classes; several paths are separated by colon. 
\rjc needs this option to correctly locate RJava files and recognize their packages. 
For example, org.yourcomp.HelloWorld should be located at 
PATH/org/yourcomp/HelloWorld.java. 
\item
\textbf{SOURCE\textunderscore FILE(S) [OPTIONAL] \\}
name one SOURCE\textunderscore FILE or several SOURCE\textunderscore FILES
to be compiled. \rjc will also compile any classes referenced and used. 
When left blank, \rjc will compile all the RJava files under the defined `dir'. 
\item
\textbf{-l SOURCE\textunderscore LIST\textunderscore FILE [OPTIONAL] \\}
name a SOURCE\textunderscore LIST\textunderscore FILE. 
A SOURCE\textunderscore LIST\textunderscore FILE should be a pure ascii file, 
and have one source file name per line. \rjc will compile all the denoted source
files in the SOURCE\textunderscore LIST\textunderscore FILE. 

\end{itemize}

\subsubsection*{Output options}
\begin{itemize}
\item
\textbf{-o OUTPUT [OPTIONAL] \\}
the final binary will be named as OUTPUT. This option will affect the GNU
Makefile that \rjc generates. \emph{By default}, OUTPUT is named after
the class name which the main method is located in. When \rjc meets
several main methods during compilation, OUTPUT will be set to
the first class that contains a main method, and \rjc will report warnings
for the following encounters of main methods. When there is no main 
methods in the source files (usually when compiling a library), OUTPUT
will be set as `lib'. 
\item
\textbf{-outdir OUTDIR [OPTIONAL] \\}
the target files will be generated to OUTDIR. \emph{By default}, OUTDIR
is the `output' directory under current working directory. This option is
currently \emph{ignored}. 
\item
\textbf{-m [OPTIONAL] \\}
setting this flag will mute console output of \rjc. It is usually set when
chaining a large number of compilations (e.g. unit testing). 
\emph{By default}, this flag is unset. 
\item
\textbf{-dt [OPTIONAL] \\}
setting this flag will facilitate debugging on target code. 
When C is the target, this flag will additionally set `-g' as gcc flags when 
generating GNU Makefile. \emph{By default}, this flag is unset. 
\end{itemize}

\subsubsection*{Target options}
\begin{itemize}
\item
\textbf{-m32 [OPTIONAL]\\}
setting this flag will instruct the target code to be compiled for
32 bits address. When C is the target, this flag will set `-m32'
as gcc flags when generating GNU Makefile. 
\emph{By default}, this flag is unset. 
\item
\textbf{-host\textunderscore os OS [OPTIONAL]\\}
the target code will be compiled for and execute on the
named OS. 
Part of RJava runtime contains OS-dependent code. 
Currently only MAC OS X (`mac') and Linux (`linux') are
supported. \emph{By default}, host OS is `mac'. 
\end{itemize}

\subsubsection*{Target-specific options}
The \rjc allows different targets for RJava to compile into. 
Options that start with
`-target' is considered as target-specific options, and will be
processed by different target code generators. 
Note that current implementation only targets C. 

\paragraph*{Target-specific options for targeting C}

\begin{itemize}
\item
\textbf{-target:mm=VAL [OPTIONAL]\\}
name VAL as the memory management scheme for 
the target code. Currently, available options are 
using hans-boehm conservative GC (`boehm'),
and using default malloc (`malloc'). The latter makes
the generated code leak memory since RJava does 
not provide mechanisms to explicitly free memory;
however, it can be useful when allocated memory tends
to be permanent. A third option is provided as 
`boehm-prebuilt', which uses the static library built beforehand
from hans-boehm GC (currently 4 versions of the static libraries
exist, 32/64bits version for Linux/Mac). 
This option is intended to save
the time of building hans-boehm GC from sources during testing. 
\emph{By default}, `boehm-prebuilt' is used. 

\end{itemize}

\subsection{The `rjc' script}

Since setting environment path options for the \rjcfull is 
tedious and error-prone, and using \rjc jar when current
working directory is not the \rjc root directory makes the situation 
even more frustrated, 
a `rjc' perl script is provided under the root directory for
improved usability. 

The `rjc' script will automatically set
those environment paths, and invoke the \rjcfull under 
\$root/build directory. If the build directory does not exist, 
it will build the compiler first. 
All options other than environment setting options can be passed
to the script, which will relay to the \rjcfull. 
The script also works when current working directory is not the
\rjc root directory. 
Be sure to keep the `rjc' script \emph{unmoved} in the \rjc root
directory. 

It is always preferred to use the `rjc' script rather than directly using
the \rjc jar. 

In next section, we will describe the implementation of the \rjcfull,
which is intended for RJava developers. 

%%%%%%%%%%%%%%%%%%
% RJava dev manual      %
%%%%%%%%%%%%%%%%%%

% 3. RJava Compiler Implementation
\section{\rjcfull Implementation}

The \rjcfull is an RJava ahead-of-time (AOT) compiler written in Java. 
Currently it targets only C, translating RJava into C. 
However, its design bares multiple targets, 
and more targets can be added in the future while reusing most of the
compiler features. 

The \rjcfull achieves the following major tasks:
\begin{itemize}
\item utilizing Soot (McGill Sable Group) to parse RJava source files and 
generating Jimple AST
\item performing RJava-level analysis and optimizations (leaving target-level
optimizations to backends)
\item generating target code and runtime
\end{itemize}

\subsection{Detailed Workflow}

This subsection describes the detailed procedure of translating 
RJava into target code. For easy matching the description here 
with the actual code, we also reference the key classes and
methods. 

\begin{enumerate}
\item
\textbf{Processing arguments and forming compilation task\\}
The main method of \rjcfull
(\javacode{org.rjava.compiler.RJavaCompiler.main()})
starts by processing the arguments and 
setting options/flags one by one. The source files to compile
will be extracted and encapsulated as a CompilationTask
(\javacode{org.rjava.compiler.CompilationTask})
no matter they are denoted as a file of source list, 
a set of source file names, or all sources under a certain directory. 
CompilationTask store source class names and the
classpath of the classes (fed as the `-dir' argument). 

\item
\textbf{Instantiating compiler instance\\}
A RJava Compiler instance will be instantiated, whose
constructor will take the CompilationTask as its
argument. After the instantiation of a 
\javacode{org.rjava.compiler.RJavaCompiler} instance
(singleton pattern), 
\javacode{init()}, \javacode{compile()}, and 
\javacode{finish()} will be executed sequentially. 
Each of them will be discussed later. 

\item
\textbf{Initializing the compiler\\}
The source code parsing (done by Soot) 
and all the static analyses happen at this step. 
  \begin{enumerate}
  \item
  \textbf{Initializing code generator\\}
  CodeGenerator (\javacode{org.rjava.compiler.target.CodeGenerator}, 
  abstract class) is the part to generate the target code. It needs to
  be initialized according to some arguments processed earlier
  (\javacode{CodeGenerator.init()}).
  \item
  \textbf{Initializing SemanticMap\\}
  SemanticMap (\javacode{org.rjava.compiler.semantics.SemanticMap})
  stores semantic information about current compiling program
  and different analyses. This step accomplishes the following tasks:
    \begin{itemize}

    \item
    \textbf{Parsing sources and generating AST\\}
    Soot is wrapped as \javacode{org.rjava.compiler.semantics.SootEngine}.
    In this step, Soot parse all the related classes 
    (including the imported classes). 
    Note that we do not
    actually `run' Soot and its optimizations, we simply use
    Soot to parse sources and use its Jimple AST. \emph{All} the analyses and
    compilation in \rjcfull are based on Jimple AST. 
    
    \item
    \textbf{Instantiating \rjc class representation\\}
    After last step, we have Soot representation of 
    classes (\javacode{soot.SootClass})
    /methods (\javacode{soot.SootMethod})/etc. 
    To make analyses convenient, we wrap those into \rjc 
    representations (see package 
    \javacode{org.rjava.compiler.semantics.representation.*}). 
	We only make our own representations for classes, 
	methods, 	statements and invoke expressions; for the rest,
	we use Soot representations. 
    Classes that are not denoted as source files but are 
    referenced will also have their \rjc representations
    and \rjc will generate code for them (as long as they are
    application classes, see \javacode{SemanticMap.isApplicationClass()}).

    \item
    \textbf{Removing generic bridging methods\\}
    Soot may generate bridge methods for generics, which are
    two methods that have same signature but different
    return types. One has the actual method body, and the
    other simply calls that method. 
    In \rjc code base, these methods are referred as 
    `twin methods' 
    (see \javacode{org.rjava.compiler.semantics.representation.RMethod} for more details)
    . We do not need bridging methods, and 
    methods with the same signature complicates demangling. 
    Thus we remove the bridging methods. 
    
    \item
    \textbf{Running multiple analyses\\}
    All static analyses that \rjc needs are done here. 
    The implementation of analyses inherits from
    CompilationPass (\javacode{org.rjava.compiler.pass.CompilationPass}).
    Note that one analysis may do several passes internally. 
    The analyses include 
    class hierarchy analysis (\javacode{ClassHierarchyPass}),
    call graph (\javacode{CallGraphPass}), 
    class initialization dependency (\javacode{DependencyGraphPass}),
    points-to analysis (\javacode{PointsToAnalysisPass}),
    constant propagation (\javacode{ConstantPropagationPass}). 
    In addition, a restriction pass is executed here to apply
    RJava's restriction model (e.g. propagating restrictions
    to nested classes). 
    
    \end{itemize}
  \end{enumerate}

\end{enumerate}

\subsection{Codebase Overview}
\subsection{Unit Tests}

% 4. RJava Compiler Details
\section{\rjcfull Details}

\subsection{Magic/Unboxed Types}
\subsection{java.lang.* Package}
\subsection{\rjcfull AST}
\subsection{Analysis and Optimization passes}
\subsection{\rjcfull Targets}

%%%%%%%%%%%%%%%%%%
% MMTk user manual    %
%%%%%%%%%%%%%%%%%%
% 5. MMTk
\section{MMTk/RJava Manual}

\subsection{Unofficial Changes}
\subsection{MMTk-VM Interface}

\end{document}
