\documentclass[12pt]{article}
\usepackage{xspace}
\title{RJava User and Developer Manual}

\author{Yi Lin\\yi.lin@anu.edu.au}

\begin{document}
\newcommand{\rjc}{RJC\xspace}
\newcommand{\rjcfull}{RJava Compiler\xspace}
\maketitle

\begin{abstract}
% rjava
RJava is a restricted subset of the Java language 
with low-level extensions
that allow access to hardware and operating system. 
% rjava benefits
RJava utilizes the same syntax as Java, and consequently
inherits benefits from Java such as type safety, 
various software engineering tools and productivity. Futhermore, 
by restrictions, RJava is a fully static language with closed world
assumption. Thus it requires a much more succinct runtime, and
is well suitable for aggressive static compilation and optimizations. 
% rjava use
RJava is designed to be an implementation language for virtual machine
construction (and more broadly for system programming). 

This manual describes the language and its 
current implementation--the \rjcfull~(\rjc). It is intended for RJava users
and developers who are willing to contribute. This manual will be
maintained to keep pace with the \rjc code base. 
\end{abstract}
\clearpage

\setcounter{secnumdepth}{5}
\setcounter{tocdepth}{5}
\tableofcontents 
\clearpage

%%%%%%%%%%%%%%%%%%
% RJava user manual     %
%%%%%%%%%%%%%%%%%%

% 1. RJava Basics
\section{RJava Basics}

\subsection{Relation between RJava and MMTk/vmmagic}

\subsection{RJava Core Restrictions}

\subsection{RJava Extensions}

% 2. RJava Compiler Tools
\section{\rjcfull Tools}

\subsection{Command Line Options}

%%%%%%%%%%%%%%%%%%
% RJava dev manual      %
%%%%%%%%%%%%%%%%%%

% 3. RJava Compiler Implementation
\section{\rjcfull Implementation}

\subsection{Codebase Overview}
\subsection{Basic Workflow}
\subsection{Unit Tests}

% 4. RJava Compiler Details
\section{\rjcfull Details}

\subsection{Magic/Unboxed Types}
\subsection{java.lang.* Package}
\subsection{\rjcfull AST}
\subsection{Analysis and Optimization passes}
\subsection{\rjcfull Targets}

%%%%%%%%%%%%%%%%%%
% MMTk user manual    %
%%%%%%%%%%%%%%%%%%
% 5. MMTk
\section{MMTk/RJava Manual}

\subsection{Unofficial Changes}
\subsection{MMTk-VM Interface}

\end{document}