\documentclass[12pt]{article}
\usepackage{xspace}
\usepackage{listings}
\usepackage{color}
\usepackage{hyperref}
\usepackage{amsmath}
\usepackage{ulem}

\title{RJava User and Developer Manual}

\author{Yi Lin\\yi.lin@anu.edu.au}

\begin{document}

\normalem

% macros
\newcommand{\rjc}{RJC\xspace}
\newcommand{\rjcfull}{RJava Compiler\xspace}

% java code format
% use \begin{lstlisting} your code \end{lstlisting}
\definecolor{dkgreen}{rgb}{0,0.6,0}
\definecolor{gray}{rgb}{0.5,0.5,0.5}
\definecolor{mauve}{rgb}{0.58,0,0.82}

\lstset{frame=tb,
  language=java,
  aboveskip=3mm,
  belowskip=3mm,
  showstringspaces=false,
  columns=flexible,
  basicstyle={\small\ttfamily},
  numbers=none,
  numberstyle=\tiny\color{gray},
  keywordstyle=\color{blue},
  commentstyle=\color{dkgreen},
  stringstyle=\color{mauve},
  breaklines=true,
  breakatwhitespace=true
  tabsize=3
}

\lstset{
  language=c,                % choose the language of the code
  numbers=left,                   % where to put the line-numbers
  stepnumber=1,                   % the step between two line-numbers.        
  numbersep=5pt,                  % how far the line-numbers are from the code
  backgroundcolor=\color{white},  % choose the background color. You must add \usepackage{color}
  showspaces=false,               % show spaces adding particular underscores
  showstringspaces=false,         % underline spaces within strings
  showtabs=false,                 % show tabs within strings adding particular underscores
  tabsize=2,                      % sets default tabsize to 2 spaces
  captionpos=b,                   % sets the caption-position to bottom
  breaklines=true,                % sets automatic line breaking
  breakatwhitespace=true,         % sets if automatic breaks should only happen at whitespace
  title=\lstname,                 % show the filename of files included with \lstinputlisting;
}

\newcommand{\javacode}[1]{{\lstinline[language=java]@#1@}}
\newcommand{\ccode}[1]{{\lstinline[language=c]@#1@}}

% cmd
\definecolor{MyGray}{rgb}{0.78,0.78,0.78}
\makeatletter\newenvironment{cmd}{%
   \newline
   \sffamily
   \noindent
   \begin{lrbox}{\@tempboxa}
   \begin{minipage}{0.8\columnwidth}}{\end{minipage}\end{lrbox}%
   \colorbox{MyGray}{\usebox{\@tempboxa}}
   \newline
}\makeatother

\newcommand{\msout}[1]{\text{\sout{\ensuremath{#1}}}}

\maketitle

\begin{abstract}
% rjava
RJava is a restricted subset of the Java language 
with low-level extensions
that allow access to hardware and operating system. 
% rjava benefits
RJava utilizes the same syntax as Java, and consequently
inherits benefits from Java such as type safety, 
various software engineering tools and productivity. Futhermore, 
by restrictions, RJava is a fully static language with closed world
assumption. Thus it requires a much more succinct runtime, and
is well suitable for aggressive static compilation and optimizations. 
% rjava use
RJava is designed to be an implementation language for virtual machine
construction (and more broadly for system programming). 

This manual describes the language and its 
current implementation--the \rjcfull~(\rjc). It is intended for RJava users
and developers who are willing to contribute. This manual will be
maintained to keep pace with the \rjc code base. 
\end{abstract}
\clearpage

\setcounter{secnumdepth}{5}
\setcounter{tocdepth}{5}
\tableofcontents 
\clearpage

%%%%%%%%%%%%%%%%%%
% RJava user manual     %
%%%%%%%%%%%%%%%%%%

% 1. RJava Basics
\section{RJava Basics}

This section describes about RJava's restrictions and extensions, 
which make RJava differentiate from the vanilla Java. 

\paragraph*{What is RJava?}
\begin{quotation}
Generally speaking, unless what is \emph{restricted} and 
what is \emph{extended},
the other part of RJava remains the same as Java (
syntactically and semantically) . Besides,
the restrictions and the extensions are also
compliant with Java \emph{syntax}. 
\end{quotation}

\noindent
Thus, Java compilers and static analyzers can 
parse RJava code (they are syntactically same).
Java runtimes can
execute RJava programs as well; However, Java runtimes
will not be able to recognize RJava
specific restrictions and extension, thus executing
RJava with Java runtimes may result in wrong results, which
is not encouraged. 

RJava is also compliant with existing Java editors and IDEs. 
When using Java IDE to edit RJava code, extensions and restrictions
need to be imported and also try avoid using execution from IDE. 

\subsection{RJava Restrictions}

One important feature about RJava is that all its restrictions 
are formalized and any RJava code need to declare the restrictions
it complies with. The benefit is that both the developers
and the RJava compilers will be able to see the declaration
and ensure the restrictions are met. 

\subsubsection{Restriction Rules and Rulesets}

RJava restrictions are defined as Java annotations, and they are
annotated by @\javacode{RestrictionRule (org.rjava.restriction.RestrictionRule)}. 
The following is how @\javacode{NoException} restriction is defined:

\begin{lstlisting}
@RestrictionRule
public @interface NoException{
}
\end{lstlisting}

Restriction rules can be used to annotate classes or methods, in the same way as
Java annotations are used. However, it is \emph{not} encouraged to directly use
restriction rules. The preferred way is to form a @\javacode{RestrictionRuleset} by
a set of chosen @\javacode{RestrictionRule}, and annotate a certain scope of code 
with such @\javacode{RestrictionRuleset}. 

One example is the MMTk scope and its restriction @\javacode{MMTk}. The following
ruleset indicates that any code within @\javacode{MMTk} needs to obey 
@\javacode{RJavaCore} ruleset (any RJava code need to include this ruleset, it will be
described in next subsection) as well as two additional restrictions
(@\javacode{NoRuntimeAllocation} and @\javacode{Uninterruptible}). The
declaration is as below:

\begin{lstlisting}
@RestrictionRuleset

@RJavaCore
@NoRuntimeAllocation
@Uninterruptible
public @interface MMTk{
}
\end{lstlisting}

\subsubsection{Restriction Model}
Restrictions follow those rules:

\begin{itemize}

\item 
Restriction rules and restriction rulesets apply to scopes of
classes and methods to declare restrictions, \emph{not} to fields. 
\item
Restriction rules and rulesets may apply to another restriction ruleset,
to indicate that the latter ruleset \emph{includes} those rules and rulesets.
This may happen iteratively. 
\item
Any RJava class needs to be restricted by @\javacode{RJavaCore}, or restricted
by a ruleset that (iteratively) includes @\javacode{RJavaCore}.
\item
A scope restricted by a restriction ruleset is restricted by every single
restriction rule within that ruleset. 
\item
If a class is restricted by Rule A, all its methods are restricted by Rule A 
unless the method is restricted otherwise 
(particularly @\javacode{Uninterruptible} vs. @\javacode{Interruptible}). 
\item
Restrictions on a class do \emph{not} affect its child classes. Restrictions on a method
do \emph{not} affect its overriding methods. 
\item
Restrictions on a class \emph{affect} its nested classes. As a result, 
those restrictions also \emph{affect} the methods of the nested class. 

\end{itemize}

\subsubsection{RJavaCore Ruleset}

@\javacode{RJavaCore} declares the basic restrictions that any RJava-compliant code
need to follow. It declares as:

\begin{lstlisting}[language=java]
@RestrictionRuleset

@NoDynamicLoading
@NoReflection
@NoException
@NoCastOnMagicType
@NoExplicitLibrary
@NoEnum
public @interface RJavaCore {
}
\end{lstlisting}

\noindent
Specifically, 
\begin{itemize}

\item
@\javacode{NoDynamicLoading}: 
the code is not allowed to dynamically load classes. To forbid this, 
\javacode{java.lang.Class.forName()} and 
\javacode{java.lang.Classloader.loadClass()}
are not allowed. Also inheriting from {java.lang.Classloader} is also not allowed. 

\item
@\javacode{NoReflection}: 
the code is not allowed to use any reflection-based feature. To forbid this,
any method that may return \javacode{java.lang.Class} is not allowed. 
@\javacode{NoExplicitLibrary} already forbids the use of 
\javacode{java.lang.reflect} package. Futhermore, 
\javacode{java.lang.Object.getClass()} 
is not allowed (\javacode{.class} syntax is also not allowed, since 
on bytecode level, it gets translated into \javacode{Object.getClass()}. )

\item
@\javacode{NoException}:
the code is not allowed to throw exceptions. To forbit this, 
\javacode{throws} in method declarations and also
\javacode{throw} statement are not allowed. Catching blocks
are allowed in order to maintain correct syntax when using
library methods, however, there will be \emph{no} exception thrown and 
catching blocks will \emph{not}
get executed in any case. It is preferred to write such code
as below:
\begin{lstlisting}[language=java]
Object lock = new Object();
try {
  lock.wait();
} catch (InterruptedException ignore) {}
\end{lstlisting}

\item
@\javacode{NoCastOnMagicType}:
the code is not allowed to do any type casting when 
\emph{at least one} side is RJava magic types. 
RJava magic types reside in \javacode{org.vmmagic.unboxed}
package, including 5 unboxed magic types (\javacode{Address}, 
\javacode{Extent}, \javacode{ObjectReference}, 
\javacode{Offset}, \javacode{Word}) and their array counterparts. 
Allowed type casting for magic types can be done via
provided methods in those classes 
(more will be discussed in the section about extensions). 

\item
@\javacode{NoExplicitLibrary}:
the code is not allowed to import any java library. However,
classes in the \javacode{java.lang.*} package are implicitly 
imported to any Java code, and they are entangled
with Java syntax. RJava \emph{allows} the use of 
\javacode{java.lang.*}. Current \rjcfull has only implemented
a subset of the \javacode{java.lang.*} package, 
more complete implementation will be done in the future. 

\item
@\javacode{NoEnum}:
the code is not allowed to use enumerate type. Enumerate
type is forbidden since enabling full features of 
Java \javacode{enum} would highly involve with dynamic behaviors
and extensive use of library methods from 
\javacode{java.lang.Enum}. We designed RJava to be 
a \emph{fully static} language, thus we forbid the 
use of enumerate type. 

\end{itemize}

These core restrictions define the RJava language. Any valid
RJava code needs to declare their compliance with @\javacode{RJavaCore}. 

By these restrictions, RJava $1)$ allows closed world assumption, 
which helps aggressive static compilation, $2)$ supports little dynamic behaviors
and bares very succinct execution runtime, which makes RJava possible to 
run with limited hardware resources, and $3)$ is a simple yet still expressive
language. 

\subsubsection{Other Predefined Restrictions}
RJava defines the above \emph{restriction model}. 
RJava encourages its users to utilize this model and, if favorable, add 
their own restriction rules and rulesets to more precisely describe
restrictions on their own scope. However, RJava predefined a set
of restrictions. Some are already mentioned in the sections above. 
These predefined restrictions are subject to change, check 
\javacode{org.rjava.restriction.rules} package in the code base for
the latest information. 

\subsection{RJava Extensions}

In order to undertake system programming task, RJava introduces
extensions to allow efficient access to hardware and operating 
systems. RJava extensions include $1)$ the \javacode{org.vmmagic}
package (by Daniel et al.) that allows memory/address representation and operations,
and $2)$ the \javacode{org.rjava.osext} package that provides 
access to operating systems (including some system calls). 

\subsubsection{org.vmmagic}

The \javacode{org.vmmagic} package provides 5 unboxed magic types
to describe pointer-alike types and their operations. Though implementation
may vary, `unboxed' means those types are not normal RJava objects,
and they are more like primitive types though there are methods declared 
for each type. 
These magic types share the same length as the pointer length on the target machine:

\begin{itemize}
\item
\javacode{ObjectReference}: can be cast from and to an RJava object. 
\emph{No} arithmetic, load/store or comparison operations are supported. 
\javacode{ObjectReference} can be cast to \javacode{Address} for 
further operations (unsafe). 

\item
\javacode{Address}: used as a pointer-alike type. 
Arithmetic, load/store and comparison operations are supported. 

\item
\javacode{Extent}: used to describe size in bytes (unsigned positive value)
Arithmetic and comparison operations are supported. 

\item
\javacode{Offset}: used to describe offset in bytes (signed). 
Arithmetic and comparison oeprations are supported. 

\item
\javacode{Word}: used as a pointer-sized integer. 
Arithmetic, comparison and bit-wised operations are supported. 
Can be cast from and to the other 4 magic types. 
\end{itemize}

\noindent
The above unboxed magic types each have an array counterpart, namely
\javacode{ObjectReferenceArray}, \javacode{AddressArray}, 
\javacode{ExtentArray}, \javacode{OffsetArray},
\javacode{WordArray}. 
They provide \javacode{create()}, 
\javacode{get()/set()} and \javacode{length()} operations. 
Arrays of those magic types should \emph{only} be created by using these types.

Besides magic types, \javacode{org.vmmagic} provides `pragma'
for supplying information to the compiler. Useful 
`pragmas' include @\javacode{Inline}, @\javacode{NoInline}, 
@\javacode{NoBoundsCheck}, @\javacode{NoNullCheck}, etc. Some `pragmas' 
from \javacode{org.vmmagic} are now considered as an 
RJava @\javacode{RestrictionRule} such as @\javacode{Uninterruptible}, 
and some are very specific to Java 
and not appliable for RJava thus will be deleted. This part is still a draft, and
subject to change. 

\subsubsection{org.rjava.osext}

The \javacode{osext} extension allows access to operating systems. Currently
\javacode{osext} only includes a minimum set of methods for implementing
a memory manager (MMTk), and it \emph{will} expand during further development. 

\begin{itemize}
\item
\javacode{OSConcurrency}: concurrency/threading related methods. 
  \begin{itemize}
  \item \javacode{void mutexLock/Unlock(Object lock)}: 
  provides an alternative to using synchronization on an RJava object. 
  \item \javacode{void threadSuspend/Resume(Thread t)}:
  suspending an RJava thread (unsafe)
  \end{itemize}

\item
\javacode{OSMemory}: memory related methods. 
  \begin{itemize}
  \item \javacode{Address malloc(int size)}:
  allocates raw memory
  \item \javacode{Address mmap(Address start, Extent length, int protection, int flags, int fd, Offset offset)}:
  calls mmap system call
  \item \javacode{int mprotect(Address start, Extent length, int prot)}:
  calls mprotect system call
  \item \javacode{Address memset(Address start, int c, Extent length)}:
  calls memset system call
  \end{itemize}

\item
\javacode{OSNative}: other methods that calls to native code
  \begin{itemize}
  \item \javacode{int errno()}: a wrap of errno() in errno.h
  \item \javacode{String strErrno()}: a warp of strerror()
  \item \javacode{double random()}: generates a random double between 0 and 1
  \end{itemize}

\end{itemize}

\subsection{Relation between RJava and MMTk/JikesRVM}

The RJava project was motivated by trying improve the portability of Memory Management
ToolKit (MMTk). MMTk started with JikesRVM project, and serves as its memory manager. 
The same as the rest part of JikesRVM, MMTk is also written in a variant of Java 
(ad-hoc restrictions with \javacode{org.vmmagic} extensions). 
Though MMTk was designed to be a portable language-agnostic memory manager, 
its portability was never a success. 

One of the main reasons that constrain its portability is the portability of the language it
is written in. On one hand, such Java variant is specially tailored, and requires support from 
its hosting runtime. It is no longer a `write once run everywhere' Java program, and
it cannot execute on a stock Java VM. Whoever wants to host MMTk needs to find a way
to execute the Java variant that MMTk is written in. 
On the other hand, MMTk is written in such an Java variant, and 
there is an inneglectable performance cost to integrate MMTk(Java) with a hosting runtime 
written in C/C++ which is the most common case. Past experiences (VMKit, Rotor, GHC)
of porting MMTk took an approach of ahead-of-time (AOT) compiling MMTk to native codes, which overcame
these two obstacles. However, \emph{none} of their approaches is general or reusable, and each
of them took a great effort. Furthermore, in term of performance, 
\emph{none} of the experiences was reported as a success (meanwhile MMTk hosted by JikesRVM
achieves excellent performance). 

Thus the idea of formalizing the Java variant that MMTk is written in 
and providing an AOT compiler to lower such language for effective
integration with low-level C/C++ code becomes a promising solution
to MMTk portability issues. This is one part of the motivations. 

Another important motivation that evolves the RJava project is the clear benefits
from using a higher-level language for virtual machine implementation. 
Compared with traditional approach of using C/C++, higher-level languages
provide benefits in safety and productivity, both of which are valuable in 
VM construction. However, existing experiences of using high-level languages
for VM construction (including JikesRVM and MMTk) all formed their own
variant of such high-level languages, which not only results in inability of code
reusing but also practically introduces metacircularity issues. As we 
already found out, metacircularity should be orthogonal to using high-level languages;
however, in practice so far, whenever high-level languages are used in virtual machine 
construction, the chosen implementation language is always the target language of 
the VM. Thus this project not only wants to settle the reusability/portability issues
but also wants to create a general implementation language
that can deliver high-level language benefits without falling into metacircular traps. 

In short, RJava originates from the ad-hoc coding pattern used in MMTk, and absorbs
the \javacode{org.vmmagic} package which is also used in MMTk. However, it evolves
to be a more general implementation languages to deliver higher-level language benefits
as well as good performance. 

In next section, we wil describe the usage of the \rjcfull, current implementation of RJava. 

\clearpage

% 2. RJava Compiler Tools
\section{\rjcfull Tools}

The \rjcfull (\rjc) is an ahead-of-time compiler for RJava. It parses RJava programs, checks
restriction compliance and translates into the target C code. Then the C backends could further
compile the generated code into binary or other form of instructions (e.g. LLVM IR). 

\subsection{Building \rjc}

The \rjcfull is written in Java. An ant build file is provided for automatic building. 
\begin{enumerate}
\item Go to \rjc root directory (\$root). 
\item Use
\begin{cmd}
ant -f mybuild.xml
\end{cmd}
After the building succeeds, 
in \$root/build, the following files
can be found:
  \begin{itemize}
  \item components/: a copy of external Java archives that \rjc uses;
  \item rjava\textunderscore ext/ : a copy of RJava extensions source file;
  \item rjava\textunderscore rt/ : a copy of RJava runtime source file;
  \item rjc.jar: the executable \rjcfull archive. 
  \end{itemize}
\end{enumerate}

\subsection{RJava Helloworld}

In this subsection, we show an example of how to compiling
the simplest RJava hello world into executable. 

\begin{enumerate}
\item Type in the source code as below, save it as `HelloWorld.java'. 
For simplicity, we put the source code in the same directory as rjc.jar
(under \$root/build). 
Note that RJava uses the same file name extension as Java (.java). 

\begin{lstlisting}[language=java]
// HelloWorld.java
import org.rjava.restriction.ruleset.RJavaCore;

@RJavaCore
public class HelloWorld {
  public static void main(String[] args) {
    System.out.println("Hello RJava");
  }
}
\end{lstlisting}

\item Compiling RJava into C. 
\begin{cmd}
java -cp rjc.jar:rjava\textunderscore rt/:. org.rjava.compiler.RJavaCompiler \\
-rjava\textunderscore ext rjava\textunderscore ext/ \\
-rjava\textunderscore rt rjava\textunderscore rt/ \\
-soot\textunderscore jdk components/soot/ \\
-dir . \\
HelloWorld.java
\end{cmd}
This will invoke \rjcfull (the arguments passed to \rjc will be discussed later, 
a script is also provided to simplify arguments needed). 
By default, the generated code locates in output directory of current directory. 
The RJava runtime sources will also be copied to the output directory, so they
can be compiled along with the generated sources. And
a GNU Makefile will be generated for compiling C into binary. 

\item Compiling C into binary.
\begin{cmd}
cd output; make
\end{cmd}
By default, the Makefile will use `gcc' as the C backend to compile the generated
C files. The output binary in this case will be `HelloWorld' (named after the class
where main method is found). 

\end{enumerate}

\subsection{Full Command Line Options}

The basic use of the \rjcfull is
\begin{cmd}
java -cp rjc.jar:rjava\textunderscore rt/:. org.rjava.compiler.RJavaCompiler\newline
[-option value] [source file(s)]...
\end{cmd}

\noindent
Here lists all the command line options of the \rjcfull. 

\subsubsection*{Environment setting options}
\emph{(Note: the`rjc' script will free you from setting these options manually. )}
\begin{itemize}
\item
\textbf{-rjava\textunderscore ext PATH [REQUIRED] \\}
set PATH as the directory for RJava extension source files. PATH should contain
org/rjava/osext, org/rjava/restriction and org/vmmagic. \rjc requires this option 
to locate those extensions during parsing RJava sources.
\item
\textbf{-rjava\textunderscore rt PATH [REQUIRED] \\}
set PATH as the directory for RJava runtime source files. PATH should contain
the implementation of \javacode{java.lang.*} and other runtime source files. 
\rjc will copy the contents in this directory to the output directory, allowing
the runtime to be compiled and linked with RJava programs. 
\item
\textbf{-soot\textunderscore jdk PATH [REQUIRED] \\}
set PATH as the directory for the JDK jars that Soot needs. PATH should contain
two JDK jars, jce.jar and rt.jar. \rjc uses Soot to parse RJava source files, 
and Soot requires those two jars to locate Java's library. 
\end{itemize}

\subsubsection*{Source file options}
\begin{itemize}
\item
\textbf{-dir PATH1:PATH2:... [REQUIRED] \\}
set PATH as the classpath of RJava classes; several paths are separated by colon. 
\rjc needs this option to correctly locate RJava files and recognize their packages. 
For example, org.yourcomp.HelloWorld should be located at 
PATH/org/yourcomp/HelloWorld.java. 
\item
\textbf{SOURCE\textunderscore FILE(S) [OPTIONAL] \\}
name one SOURCE\textunderscore FILE or several SOURCE\textunderscore FILES
to be compiled. \rjc will also compile any classes referenced and used. 
When left blank, \rjc will compile all the RJava files under the defined `dir'. 
\item
\textbf{-l SOURCE\textunderscore LIST\textunderscore FILE [OPTIONAL] \\}
name a SOURCE\textunderscore LIST\textunderscore FILE. 
A SOURCE\textunderscore LIST\textunderscore FILE should be a pure ascii file, 
and have one source file name per line. \rjc will compile all the denoted source
files in the SOURCE\textunderscore LIST\textunderscore FILE. 

\end{itemize}

\subsubsection*{Output options}
\begin{itemize}
\item
\textbf{-o OUTPUT [OPTIONAL] \\}
the final binary will be named as OUTPUT. This option will affect the GNU
Makefile that \rjc generates. \emph{By default}, OUTPUT is named after
the class name which the main method is located in. When \rjc meets
several main methods during compilation, OUTPUT will be set to
the first class that contains a main method, and \rjc will report warnings
for the following encounters of main methods. When there is no main 
methods in the source files (usually when compiling a library), OUTPUT
will be set as `lib'. 
\item
\textbf{-outdir OUTDIR [OPTIONAL] \\}
the target files will be generated to OUTDIR. \emph{By default}, OUTDIR
is the `output' directory under current working directory. This option is
currently \emph{ignored}. 
\item
\textbf{-m [OPTIONAL] \\}
setting this flag will mute console output of \rjc. It is usually set when
chaining a large number of compilations (e.g. unit testing). 
\emph{By default}, this flag is unset. 
\item
\textbf{-dt [OPTIONAL] \\}
setting this flag will facilitate debugging on target code. 
When C is the target, this flag will additionally set `-g' as gcc flags when 
generating GNU Makefile. \emph{By default}, this flag is unset. 
\end{itemize}

\subsubsection*{Target options}
\begin{itemize}
\item
\textbf{-m32 [OPTIONAL]\\}
setting this flag will instruct the target code to be compiled for
32 bits address. When C is the target, this flag will set `-m32'
as gcc flags when generating GNU Makefile. 
\emph{By default}, this flag is unset. 
\item
\textbf{-host\textunderscore os OS [OPTIONAL]\\}
the target code will be compiled for and execute on the
named OS. 
Part of RJava runtime contains OS-dependent code. 
Currently only MAC OS X (`mac') and Linux (`linux') are
supported. \emph{By default}, host OS is `mac'. 
\end{itemize}

\subsubsection*{Target-specific options}
The \rjc allows different targets for RJava to compile into. 
Options that start with
`-target' is considered as target-specific options, and will be
processed by different target code generators. 

\paragraph*{Target-specific options for targeting C}

\begin{itemize}
\item
\textbf{-target:mm=VAL [OPTIONAL]\\}
name VAL as the memory management scheme for 
the target code. Currently, available options are 
using hans-boehm conservative GC (`boehm'),
and using default malloc (`malloc'). The latter makes
the generated code leak memory since RJava does 
not provide mechanisms to explicitly free memory;
however, it can be useful when allocated memory tends
to be permanent. A third option is provided as 
`boehm-prebuilt', which uses the static library built beforehand
from hans-boehm GC (currently 4 versions of the static libraries
exist, 32/64bits version for Linux/Mac). 
This option is intended to save
the time of building hans-boehm GC from sources during testing. 
\emph{By default}, `boehm-prebuilt' is used. 

\end{itemize}

\subsection{The `rjc' script}

Since setting environment path options for the \rjcfull is 
tedious and error-prone, and using \rjc jar when current
working directory is not the \rjc root directory makes the situation 
even more frustrated, 
a `rjc' perl script is provided under the root directory for
improved usability. 

The `rjc' script will automatically set
those environment paths, and invoke the \rjcfull under 
\$root/build directory. If the build directory does not exist, 
it will build the compiler first. 
All options other than environment setting options can be passed
to the script, which will relay to the \rjcfull. 
The script also works when current working directory is not the
\rjc root directory. 
Be sure to keep the `rjc' script \emph{unmoved} in the \rjc root
directory. 

It is always preferred to use the `rjc' script rather than directly using
the \rjc jar. 

In next section, we will describe the implementation of the \rjcfull,
which is intended for RJava developers. 

%%%%%%%%%%%%%%%%%%
% RJava dev manual      %
%%%%%%%%%%%%%%%%%%

% 3. RJava Compiler Implementation
\section{\rjcfull Implementation}

The \rjcfull is an RJava ahead-of-time (AOT) compiler written in Java. 
Currently it targets only C, translating RJava into C. 
However, its design bares multiple targets, 
and more targets can be added in the future while reusing most of the
compiler features. 

The \rjcfull achieves the following major tasks:
\begin{itemize}
\item utilizing Soot (McGill Sable Group) to parse RJava source files and 
generating Jimple AST
\item performing RJava-level analysis and optimizations (leaving target-level
optimizations to backends)
\item generating target code and runtime
\end{itemize}

\subsection{Detailed Workflow}

This subsection describes the detailed procedure of translating 
RJava into target code. For easy matching the description here 
with the actual code, we also reference the key classes and
methods. 

\begin{enumerate}
\item
\textbf{Processing arguments and forming compilation task\\}
The main method of \rjcfull
(\javacode{org.rjava.compiler.RJavaCompiler.main()})
starts by processing the arguments and 
setting options/flags one by one. The source files to compile
will be extracted and encapsulated as a CompilationTask
(\javacode{org.rjava.compiler.CompilationTask})
no matter they are denoted as a file of source list, 
a set of source file names, or all sources under a certain directory. 
CompilationTask store source class names and the
classpath of the classes (fed as the `-dir' argument). 

\item
\textbf{Instantiating compiler instance\\}
A RJava Compiler instance will be instantiated, whose
constructor will take the CompilationTask as its
argument. After the instantiation of a 
\javacode{org.rjava.compiler.RJavaCompiler} instance
(singleton pattern), 
\javacode{init()}, \javacode{compile()}, and 
\javacode{finish()} will be executed sequentially. 
Each of them will be discussed later. 

\item
\textbf{Initializing the compiler\\}
The source code parsing (done by Soot) 
and all the static analyses happen at this step. 
  \begin{enumerate}
  \item
  \textbf{Initializing code generator\\}
  CodeGenerator (\javacode{org.rjava.compiler.target.CodeGenerator}, 
  abstract class) is the part to generate the target code. It needs to
  be initialized according to some arguments processed earlier
  (\javacode{CodeGenerator.init()}).
  \item
  \textbf{Initializing SemanticMap\\}
  SemanticMap (\javacode{org.rjava.compiler.semantics.SemanticMap})
  stores semantic information about current compiling program
  and different analyses. This step accomplishes the following tasks:
    \begin{itemize}

    \item
    \textbf{Parsing sources and generating AST\\}
    Soot is wrapped as \javacode{org.rjava.compiler.semantics.SootEngine}.
    In this step, Soot parse all the related classes 
    (including the imported classes) and add them to
    current \javacode{CompilationTask}. 
    Note that we do not
    actually `run' Soot and its optimizations, we simply use
    Soot to parse sources and use its Jimple AST. 
    \emph{All} the analyses and
    compilation in \rjcfull are based on Jimple AST. 
    
    \item
    \textbf{Instantiating \rjc class representation\\}
    After last step, we have Soot representation of 
    classes (\javacode{soot.SootClass})
    /methods (\javacode{soot.SootMethod})/etc. 
    To make analyses convenient, we wrap those into \rjc 
    representations (see package 
    \javacode{org.rjava.compiler.semantics.representation.*}). 
	We only make our own representations for classes, 
	methods, 	statements and invoke expressions; for the rest,
	we use Soot representations. 
    Classes that are not denoted as source files but are 
    referenced will also have their \rjc representations
    and \rjc will generate code for them (as long as they are
    application classes, see \javacode{SemanticMap.isApplicationClass()}).

    \item
    \textbf{Removing generic bridging methods\\}
    Soot may generate bridge methods for generics, which are
    two methods that have same signature but different
    return types. One has the actual method body, and the
    other simply calls that method. 
    In \rjc code base, these methods are referred as 
    `twin methods' 
    (see \javacode{org.rjava.compiler.semantics.representation.RMethod} for more details)
    . We do not need bridging methods, and 
    methods with the same signature complicates name mangling. 
    Twin methods are merged into one whose
    signature is the same as defined in source code, and
    the bridging method is removed. 
    
    \item
    \textbf{Running multiple analyses\\}
    All static analyses that \rjc needs are done here. 
    The implementation of analyses inherits from
    CompilationPass\\ (\javacode{org.rjava.compiler.pass.CompilationPass}).
    Note that one analysis may do several passes internally. 
    The analyses include 
    class hierarchy analysis (\javacode{ClassHierarchyPass}),
    call graph (\javacode{CallGraphPass}), 
    class initialization dependency (\javacode{DependencyGraphPass}),
    points-to analysis (\javacode{PointsToAnalysisPass}),
    constant propagation (\javacode{ConstantPropagationPass}). 
    In addition, a restriction pass (\javacode{RestrictionPass})
    is executed here to apply
    RJava's restriction model (e.g. propagating restrictions
    to nested classes). The results of those analyses 
    can be retrieved through \javacode{SemanticMap}. 
    
    \end{itemize}
  \end{enumerate}

\item
\textbf{Compiling\\}
After going through the steps above, \rjc has parsed source code
into AST, and also collected sufficient analysis results. The actual
compilation starts from \javacode{RJavaCompiler.compile()}. 
\rjc iterates through
all the classes in current compilation task, checks their restriction compliance
and starts translation. There are also some global target-specific works that
need to be done before and after the translation such as clearing output directory
and copy runtime sources, and respectively they are done in 
\javacode{CodeGenerator.preTranslationWork()} and
\javacode{CodeGenerator.postTranslationWork()}.

  \begin{enumerate}
  \item
  \textbf{Performing pre-translation work\\}
  In this step, \rjc creates the output directory if it does not exist
  and clear the directory. This step is `global' and execute
  only once while following steps are performed for every RJava
  class. 
  
  \item
  \textbf{Checking restrictions\\}
  \rjc checks the restriction compliance of a class before starting
  to translate it. 
  \javacode{org.rjava.compiler.restriction.StaticRestrictionChecker}
  takes charge of the compliance checking. \javacode{comply()} is
  invoked with an \javacode{RClass} argument. For every restriction
  \javacode{org.rjava.restriction.rules.A} declared on such the class,
  a checking rule (\javacode{org.rjava.restriction.rules.A_CHECK}) is expected with a
  \javacode{checkClass()} and \javacode{checkMethod()} declared. 
  The restriction checker will invoke those checking methods. 
  Code violating restrictions, failing to find checking rules or 
  a class not declaring a restriction ruleset will cause an RJava warning
  (it will not stop the translating process), 
  and will be reported. Details about restriction checking will be
  covered in next section. Note that current implementation 
  \emph{ignores} restrictions on methods, and does not check them (this will be
  fixed soon). 
  
  \item
  \textbf{Translating\\}
  This step is target-specific. \javacode{CodeGenerator.translate()} is called for
  the actual code tranlation. 
  \javacode{org.rjava.compiler.target.CodeGenerator} is expected to be inherited
  for a certain target. Currently only C target is implemented. 
  Details about C target translation will be discussed in next section. 
  
  \item
  \textbf{Performing post-translation work}
  Similar to pre-translation work, post-translation work 
  (\javacode{CodeGenerator.postTranslationWork()}) is a target-specific global step. 
  Works such as generating and copying runtime sources, generating build files for
  auto tools should be done here. 
  \end{enumerate}

\end{enumerate}

\subsection{Codebase Overview}

This part describes the overview of \rjcfull code base, 
including its main compiler project, unofficial MMTk project clone
and other miscellaneous stuff. Importing the root directory of \rjc 
code base into Eclipse IDE 
will imports the main compiler project. 

Current code base is accessible from ANU squirrel mercurial repository
\url{http://squirrel.anu.edu.au/hg/all/yilin/rjava-prototype/} 
(authorization required). 

\begin{itemize}
\item
\textbf{compiler/\\}
The main source directory of \rjc. 
  \begin{itemize}
  
  \item \javacode{org.rjava.compiler}\\
  \javacode{RJavaCompiler} main class, 
  \javacode{CompilationTask} and constants definitions
  
  \item \javacode{org.rjava.compiler.exception}\\
  \rjc warnings and errors
  
  \item \javacode{org.rjava.compiler.pass}\\
  different analysis passes
  
  \item \javacode{org.rjava.compiler.restriction}\\
  the restriction checker
  
  \item \javacode{org.rjava.compiler.semantics}\\
  \javacode{SemanticMap}, \javacode{SootEngine} 
  and other data structures
  such as \javacode{CallGraph}, \javacode{DependencyGraph}, etc.

  \item \javacode{org.rjava.compiler.semantics.representation}\\
  \rjc 's representation of class, type, method, field, local variable, 
  and annotation to wrap its Soot counterparts. Additionally, 
  \javacode{RRestriction} is defined here. 
  
  \item \javacode{org.rjava.compiler.semantics.representation.stmt}\\
  \rjc's representation of 16 different types of statements, which
  maps to different types of Jimple statements.
  
  \item \javacode{org.rjava.compiler.target}\\
  abstract \javacode{CodeGenerator} and 
  \javacode{GeneratorOptions} that fits 
  into \rjc workflow. Each target that RJava wants to translate into
  need to inherit these two classes. 
  \javacode{CodeStringBuilder} is provided here for better indentation
  of code output. 
  
  \item \javacode{org.rjava.compiler.target.c}\\
  C-specific code. \javacode{CLanguageGenerator} and
  \javacode{CLangaugeGeneratorOptions} inherits the two abstract classes
  mentioned above. \javacode{CStatements} and \javacode{CExpressions}
  generate code for finer-grained scope. \javacode{CIdentifiers} mangles
  Java names into C name. \javacode{Intrinsics} takes care of some
  C-specific intrinsics, e.g. setting RJava integer types
  to different lengths of integer types in C. 
  
  \item \javacode{org.rjava.compiler.target.c.runtime}\\
  Part of the RJava's C runtime and the GNU Makefile for the target code
  are generated during the compilation,
  which is done here. Also the magic type implementation is
  generated by reusing \rjc to compile magic type stubs; its
  implementation can be found in \javacode{MagicTypesForC}. 
  There is also an attempt to reuse \rjc to compile 
  \javacode{java.lang.*} package, but it has been put halt. 
  
  \item \javacode{org.rjava.compiler.util}\\
  data structures such as \javacode{Tree}, utility methods for easier use
  of third-party components (Soot and JGraphT). We also provide 
  special set, map and multi-value map for storing \javacode{soot.Value},
  for its special rule of equality comparison. \javacode{Statistics} utilities
  are also included here for instrumenting compiler code.
  \end{itemize}

\item
\textbf{components/\\}
Third party components used by \rjc, including common collection and IO
libraries from Apache, JGraph and JGraphT for visualizing some results of 
static analyses, and Soot for parsing RJava source code. 

\item
\textbf{document/\\}
This user and developer manual, both PDF version and tex source. 

\item
\textbf{mmtk\textunderscore standalone/\\}
An unofficial clone of Memory Management ToolKit repository. 
Minor changes from the JikesRVM MMTk are needed 
to make it compatible with RJava restrictions. Details will be discussed
in next section. This can be imported as a separate Eclipse project. 

\item
\textbf{mmtk/\\}
The symbolic link of the standalone MMTk (the above directory) along
with a minimal client required to run MMTk allocation and GC (named
as `testbed'). This can be imported as another separate Eclipse project. 

\item
\textbf{rjava\textunderscore clib\textunderscore impl/\\}
An attempt to write \javacode{java.lang.*} in RJava, and reuse \rjc
to compile it into C target. It has been stopped. 

\item
\textbf{rjava\textunderscore ext/\\}
The source files of RJava extensions, including
\javacode{org.vmmagic} and \javacode{org.rjava.osext}. 
Those source files are mostly empty stubs, and they exist
for source code completeness. 
\rjc ensures they will have actual implementation in 
the target code.

\item
\textbf{test/\\}
RJava implementation of some benchmarks, and also some
small test cases for different language features and 
optimizations. 

\item
\textbf{unittest/\\}
Unit tests for RJava implementation, including a unit test script (`unittest.pl') 
and a set of RJava unit test sources. Current unit test is
\emph{not} a thorough set covering all the features of RJava, it will 
be expanded in the future. 

\end{itemize}

\subsection{Unit Tests}

\rjcfull includes a simple framework for unit testing. 

\subsubsection{Running Unit Tests}
Run the unit tests by executing 
\begin{cmd}
unittest/unittest.pl
\end{cmd}
This script will first compile a lastest version of \rjcfull, then
detects all the source files under unittest/src (but not the abstract
base class \javacode{org.rjava.unittest.UnitTest}). It iterates through
every single unit test, invoke \rjc to compile RJava into C, then
from C into binary, and execute it. It observes the execution of every
compilation step, and also tracks the output of the binary. Execution
successes and fails are reported at the end of each unit test, and also
at the end of the script. The unit testing script is specific to C target. 

\subsubsection{Adding New Unit Tests}

Adding new unit tests is simple. Create a new RJava source file by
extending \javacode{org.rjava.unittest.UnitTest}, and save it to
unittest/unittest/src in order to be discovered by the unit testing script. 
Within the new unit test source, write your own test code, and use 
the \javacode{start()} and \javacode{check()} methods provided
by the base class for result outputs so that the testing
script can parse and recognize successes and fails. 

\clearpage

% 4. RJava Compiler Details
\section{\rjcfull Details}

This section discusses the details about the implementation of \rjcfull. 
Part of this section is \emph{specific} to current \rjc implementation 
that targets C. It aims to help developers understand
\rjc implementation. We will not emphasize this point again in 
this section. 

\subsection{\rjc Object Model}

RJava does not restrict on the use of object-oriented features of
Java, and it supports full features of class inheritance and 
interfaces. To translate the object oriented features into
non-OO target namely C, \rjc uses an object model 
similar to GtkObject of GNOME project. 

\subsubsection{Type Implementation: \ccode{RJava_Common_Class}}
\rjc defines a common type struct, i.e. \ccode{RJava_Common_Class}. Every
RJava class is matched to a type struct that embeds the common type struct. 

The common type struct (defined in \ccode{rjava_crt.h}, generated at compile-time):
\begin{lstlisting}[language=c]
struct RJava_Common_Class {
  RJava_Common_Class* super_class;
  RJava_Interface_Node* interfaces;
  pthread_mutex_t class_mutex;
};
\end{lstlisting}

For Java's common \javacode{java.lang.Object}, its type struct is:
\begin{lstlisting}[language=c]
struct java_lang_Object_class {
  RJava_Common_Class class_header;
    
  // method members
  struct java_lang_String* (*toString)();
  ...
};

struct java_lang_Object_class java_lang_Object_class_instance;
\end{lstlisting}

For a random RJava class \javacode{org.yourcomp.A}, 

\begin{lstlisting}[language=java]
package org.yourcomp;

@RJavaCore
public class A {
  int a;
  void foo(int i) {}
}
\end{lstlisting}

its type struct definition (\ccode{org_yourcomp_A.h}) is:

\begin{lstlisting}[language=c]
struct org_yourcomp_A_class {
  struct java_lang_Object_class header;
  void (*foo_int32_t)(int32_t);
}

struct org_yourcomp_A_class org_yourcomp_A_class_instance;
\end{lstlisting}

All the class instances will be initialized eagerly
in a certain order at the beginning
of main method execution 
(\ccode{rjava_class_init()} in \ccode{rjava_crt.c}, which is
generated at compile-time). Function pointer binding
is also done in \ccode{rjava_class_init()}. 

This design allows a class to access its super class, 
to declare interfaces and its own method members, 
and also to selectively override or inherit
methods and interface method from its super class. 

\subsubsection{Object Implementation: \ccode{RJava_Common_Instance}}

\rjc also defines a common instance struct, i.e. \ccode{RJava_Common_Instance}. 
Every RJava object is matched to an instance struct that embeds the common instance
struct. 

The common instance struct (defined in \ccode{rjava_crt.h}, generated at compile-time):
\begin{lstlisting}[language=c]
struct RJava_Common_Instance {
  void* class_struct;
  pthread_mutex_t instance_mutex;
  pthread_cond_t instance_cond;
};
\end{lstlisting}

For Java's common \javacode{java.lang.Object}, its instance struct is:
\begin{lstlisting}[language=c]
struct java_lang_Object {
  RJava_Common_Instance instance_header;
    
  // instance fields
  ...
};
\end{lstlisting}

For a random RJava class \javacode{org.yourcomp.A} (see the above
subsection), its instance struct definition (\ccode{org_yourcomp_A.h}) is:
\begin{lstlisting}[language=c]
struct org_yourcomp_A {
  java_lang_Object instance_header;
  
  int32_t a;
}
\end{lstlisting}

This design allows every instance to access its type information, and
to inherit fields declared by its super class. 

\subsubsection{Interface Implementation: \ccode{RJava_Interface_Node}}

An interface class in RJava is translated into a type struct and an instance
struct in the same way as normal classes. Interfaces are instantiated
for each class that implements them. For accessing interface instances
from a type, \ccode{RJava_Interface_Node} is used as a linked list to store
all the interface instances declared on this type:

\begin{lstlisting}[language=c]
struct RJava_Interface_Node {
  char* name;
  void* address;  // points to interface instance
  int interface_size;
  RJava_Interface_Node* next;
};
\end{lstlisting}

The instantiation and initialization of interface instances and 
\ccode{RJava_Interface_Node} for each type is done in 
\ccode{rjava_class_init()}. \rjc's C runtime provides methods
to help with the initialization and also with interface invocation:

\begin{itemize}
\item
\ccode{void rjava_add_interface_to_class(void* interface, int interface_size, char* name, RJava_Common_Class* klass);}
\item
\ccode{void rjava_alter_interface(void* interface, char* name, RJava_Common_Class* klass);}
\item
\ccode{void* rjava_get_interface(RJava_Interface_Node* list, char* name);}
\end{itemize}

Note that current implementation of interface makes \javacode{invokeinterface}
\emph{extremely inefficient}, and we have not implemented any optimizations
for interface. Use of interfaces should be discouraged in 
performance-centric scenarios until the
implementation gets an improvement. 

\subsubsection{Array Implementation}

RJava arrays is implemented different than Java arrays. 
RJava arrays is \emph{not} an \javacode{Object}, and it is more like
C arrays with a header. The header is required to perform
bounds checking. For the address that an array points to, 
first thing accessed is a C integer as the array length, followed
by a C long as element size. The rest is the actual array. 

This implementation is flawed, since an array does not 
carry its type information and prevents run-time type 
checking. Besides, not being an RJava \javacode{Object} is
conceptually different than Java, which is not desired. 
The array implementation will be revised. 

\subsection{Magic/Unboxed Types}

Magic types are pointer-sized integers that are used to describe
memory related operands. In RJava source code, they are expressed
as normal RJava classes; however, they are intended to 
behave more like `primitives' rather than `objects'. 

In \rjc implementation, uses of magic types in RJava code are 
translated in the same way as other types (though \rjc understands
those are magic types). And invocations for methods in magic types
are translated into a static function call (normal 
RJava application method invocation is translated a function call via
function pointer unless de-virtualized). 

\rjc provides an implementation of magic types and their methods
as a large list of C macros. So during C compilation, magic types
and their method callsites will be replaced. Magic types are each defined as:

\begin{lstlisting}[language=c]
#define org_vmmagic_unboxed_Address						uintptr_t
#define org_vmmagic_unboxed_Extent						uintptr_t
#define org_vmmagic_unboxed_ObjectReference		uintptr_t
#define org_vmmagic_unboxed_Offset						intptr_t
#define org_vmmagic_unboxed_Word							uintptr_t

#define org_vmmagic_unboxed_AddressArray			uintptr_t*
#define org_vmmagic_unboxed_ExtentArray				uintptr_t*
#define org_vmmagic_unboxed_ObjectReferenceArray		uintptr_t*
#define org_vmmagic_unboxed_OffsetArray						intptr_t*
#define org_vmmagic_unboxed_WordArray							uintptr_t*
\end{lstlisting}

Their methods are defined as (only showing a few as examples):
\begin{lstlisting}[language=c]
#define org_vmmagic_unboxed_Address_plus_int32_t(this_parameter,parameter0) \
this_parameter + parameter0
#define org_vmmagic_unboxed_Address_plus_org_vmmagic_unboxed_Offset(this_parameter,parameter0) \
this_parameter + parameter0
\end{lstlisting}

This implementation requires minimal special treatment
on magic types during translation, and brings in
low run-time overhead. 
The implementation of magic types can be generated by reusing \rjc \\
(see \javacode{org.rjava.compiler.target.c.runtime.MagicTypesForC}). 

\subsection{\javacode{java.lang.*} Package}

RJava disallows the use of Java library by 
@\javacode{NoExplicitLibrary} restriction for
its own simplicity. However, 
\javacode{java.lang.*} is entangled with 
the Java syntax and it is hardly possible to
forbid its use without crippling the language. 
Thus RJava preserves \javacode{java.lang.*}. 

\rjc implements the package in C by following
its object model. There is only one difference
between \rjc \javacode{java.lang.*} and an RJava
program that method invocations to 
\javacode{java.lang.*} are statically bounded. 
This implies that it is \emph{not} allowed to
inherit from the library classes. 

\rjc selectively picks a subset of \javacode{java.lang.*}
to implement, following these principles:
\begin{itemize}
\item enabling primitive type boxing
\item providing string type
\item providing support for threading and concurrency
\item providing empty stubs just for source completeness
\item providing basic standard output
\end{itemize}

Thus current \rjc includes
\emph{incomplete} implementation of the following
classes:

\begin{itemize}
\item \javacode{java.io.PrintStream} (only for standard output)
\item \javacode{java.lang.Boolean}
\item \javacode{java.lang.Class} (empty stub)
\item \javacode{java.lang.ClassNotFoundException} (empty stub)
\item \javacode{java.lang.Double}
\item \javacode{java.lang.Exception} (empty stub)
\item \javacode{java.lang.Float}
\item \javacode{java.lang.Integer}
\item \javacode{java.lang.InterruptedExeption} (empty stub)
\item \javacode{java.lang.Math}
\item \javacode{java.lang.NoClassDefFoundError} (empty stub)
\item \javacode{java.lang.Object}
\item \javacode{java.lang.Runnable}
\item \javacode{java.lang.String}
\item \javacode{java.lang.StringBuffer}
\item \javacode{java.lang.System}
\item \javacode{java.lang.Thread}
\item \javacode{java.lang.Throwable} (empty stub)
\end{itemize}

Current design and implementation about \javacode{java.lang} 
remains as a draft, and it might need revision in the future. 

\subsection{\rjcfull AST}

\rjc uses Soot to parse source code, and consequently also reuses
Soot's Jimple AST. Jimple's grammar is showed below (quoting
Raja Vallee-Rai's thesis):

\begin{itemize}
\item
$\begin{aligned}[t]
stmt \longrightarrow &assignStmt~|~identityStmt~|~gotoStmt~|\\
&ifStmt~|~invokeStmt~|~switchStmt~|\\
&monitorStmt~|~returnStmt~|~throwStmt~|\\
&breakpointStmt~|~nopStmt
\end{aligned}$


\item
$\begin{aligned}[t]
assignStmt \longrightarrow &\mathbf{local} = \mathbf{rvalue};~|\\
&\mathbf{field} = \mathbf{imm};~|\\
&\mathbf{local.field} = \mathbf{imm}; ~|\\
&\mathbf{local[imm]} = \mathbf{imm}; 
\end{aligned}$

\item
$\begin{aligned}[t]
identityStmt \longrightarrow &\mathbf{local} = \mathbf{this}; ~|\\
&\mathbf{local} = \mathbf{parameter_n};~|\\
&\msout{\mathbf{local} = \mathbf{exception};}\\
\end{aligned}$

\item
$gotoStmt \longrightarrow goto~label;$

\item
$ifStmt \longrightarrow if~\mathbf{conditionExpr}~goto~label;$

\item
$invokeStmt \longrightarrow invoke~\mathbf{invokeExpr};$

\item
$\begin{aligned}[t]
switchStmt \longrightarrow &lookupswitch~\mathbf{imm}\\
&\{case~\mathbf{value_1}: goto~label_1;\\
&...\\
&case~\mathbf{value_n}: goto~label_n;\\
&default: goto~defaultLabel;\};~|\\
&tableswitch~\mathbf{imm}\\
&\{case~\mathbf{low}: goto~lowLabel;\\
&...\\
&case~\mathbf{high}: goto~highLabel;\\
&default: goto~defaultLabel;\};
\end{aligned}$

\item
$\begin{aligned}[t]
monitorStmt \longrightarrow &entermonitor~\mathbf{imm};~|\\
&exitmonitor~\mathbf{imm};
\end{aligned}$

\item
$\begin{aligned}[t]
returnStmt \longrightarrow &return \mathbf{imm}~|\\
&return;
\end{aligned}$

\item
$\msout{throwStmt \longrightarrow throw~\mathbf{imm}}$

\item
$\msout{breakpointStmt \longrightarrow breakpoint;}$

\item
$nopStmt \longrightarrow nop;$
\end{itemize}

\noindent
Note: 
\begin{enumerate}
\item \textbf{bold font} means a \emph{Value} type (\javacode{soot.Value})
\item \sout{strikeout font} means rules that are not appliable to RJava. 
\item \rjc implements wrapper classes to represent different Jimple statements
(see \javacode{org.rjava.compiler.semantics.representation.stmt}), 
however, for \emph{Value} and subclasses of \emph{Value}, \rjc
directly use Soot representations. 
\end{enumerate}

\subsection{Analysis and Optimization passes}

This subsection gives detailed information about some analysis and optimization passes
in \rjc by describing them or referencing the origin of the algorithm. 
All analyses so far (inherited from \javacode{org.rjava.compiler.pass.CompilationPass})
are \emph{flow-sensitive} and \emph{path-insensitive}. 

\subsubsection{Class Initialization Order}

\rjc ahead-of-time compiles RJava code, and forbids any run-time class loading. 
Thus all the classes can be seen at compile-time, and early class initialization is desired. 

\rjc uses the algorithm from the paper `Eager Class Initialization for Java' by Dexter Kozen
and Matt Stillerman. The algorithm generates a class dependency graph; each edge A$\rightarrow$B
means the initialization of class A depends on the initialization of 
class B, and as a result, B should be initialized
before A. 

\subsubsection{Constant Propagation}

Though the C backend will do constant propagation, during translation from
RJava to C, some high-level information is lost which makes C compiler easily
miss opportunities. 

\rjc uses the algorithm from the paper `Interprocedural Constant Propagation'
by David Callaham et al. \rjc only considers \emph{numeric} constants (including integer,
floating point number, byte and character. Address-alike constants are currently
ignored. )

This optimization pass does two passes over the code to collect information, 
and performs the propagation in the end. 
It uses results from \emph{Class Hierarchy Analysis (CHA)}, 
\emph{Points-to Analysis (PTA)} and \emph{Call Graph (CG)}. Note
this optimization does not directly change AST, but store constant information
internally and provide query methods for use at code generation time. 

This optimization is optimistic, and specially takes care of situations such as:
\begin{itemize}
\item numeric \javacode{static final} fields used across different RJava files 
should be deduced as constant if possible. 
\item methods that returns constant should be propagated.
\item methods that take same constant arguments from all their callsites can be
propagated. 
\item methods that might be called by polymorphism and also match the two points
above can be propagated.  
\end{itemize}

\subsubsection{Devirtualization}

Devirtualization is the most crucial optimization in \rjc in term of performance. 
A virtualized method invocation of \javacode{obj.methodFoo()} 
(\javacode{obj} is an instance of \javacode{ClassFoo}) under \rjc object model is \\
\ccode{ ((ClassFoo_class*)(((RJava_Common_Instance*) obj) -> class_struct)) -> methodFoo()},
and a devirtualized version can be \ccode{ClassFoo_methodFoo()}. 
This not only eliminates the pointer dereferencing, but also allows inlining
from the C backend. This optimization boosts the performance by a factor of 3
when running MMTk. 

Devirtualization is not a \javacode{CompilationPass} in \rjc. It uses the results
from points-to analysis for type inference when generating code for virtual invocations. 

\subsection{Restriction Checking}

Checking the compliance of restrictions is important for RJava code robustness. 
\rjc checks the restriction at compile-time. 

\rjc assume every restriction rule (\javacode{org.rjava.restriction.rules.A})
comes with a checking rule (\javacode{org.rjava.restriction.rules.A_CHECK}),
and the checking class should contain methods with signature:
\begin{itemize}
\item \javacode{public static Boolean checkClass(RClass klass);}
\item \javacode{public static Boolean checkMethod(RMethod method);} 
(this one is not currently implemented nor invoked, \rjc only checks classes at the moment)
\end{itemize}

The implementation of checking methods is supposed to:
\begin{itemize}
\item return true if the checking passes, or false if the checking fails. 
\item create an \javacode{RJavaRestrictionViolation} object by \\
\javacode{RJavaRestrictionViolation.newRestrictionViolation()}
if checking fails, which stores information about this violation. 
\item inform the \javacode{StaticRestrictionChecker} of the violation by\\
\javacode{StaticRestrictionChecker.addRestrictionViolation()}. 
\end{itemize}

\rjc 's restriction checker will invoke such methods from these classes 
for every declared restriction rules.
If a class fails the restriction checking (either violating the restriction or 
failing to find checking classes/methods), a warning will be recorded
and reported at the end of compilation (it will not interrupt compilation). 

\subsection{C Target Translation}

The \javacode{org.rjava.compiler.target.c} package deals with 
the C target translation. Ideally this should be a package 
dedicated to C target, which means code inside should be C specific 
and not general while code outside should be general and 
target independent. 
Current implementation roughly follows this rule 
(however, not \emph{strictly}); further refactoring will fix this. 

The generated code currently complies with GNU99 standard.  
We may move to standard C99 in the future. 

\subsubsection{Code Generation}

\javacode{CLanguageGenerator} is responsible for code generation
with the help from \javacode{CStatements}, \javacode{CExpressions}
and \javacode{CIdentifiers}. 

\begin{itemize}
\item A non-interface RJava class \javacode{org.yourcomp.A}
will be translated into 3 different 
C files:
  \begin{enumerate}
  \item a header file \ccode{org_yourcomp_A.h},
  which contains the definition of the type `\ccode{struct org_yourcomp_A_class}'
  and the instance `\ccode{struct org_yourcomp_A}' from this class,
  \item a header file \ccode{org_yourcomp_A_methods.h},
  which contains the declaration of methods from \javacode{org.yourcomp.A}
  and the definition of inlining methods,
  \item a source file \ccode{org_yourcomp_A_methods.c},
  which contins the definition of non-inlining methods. 
  \end{enumerate}
Every C file will include the type or methods it used. 
The separation of method definition and type/instance definition 
can to the maximum eliminate possible circular referencing between
two different header files. 

\item An RJava interface class \javacode{org.yourcomp.InterfaceA}
will be translated into 2 different C files:
  \begin{enumerate}
  \item a header file \ccode{org_yourcomp_InterfaceA.h},
  which contains the definition of the type `\ccode{struct org_yourcomp_InterfaceA_class}',
  and the instance `\ccode{struct org_yourcomp_InterfaceA}' from this class,
  \item a source file \ccode{org_yourcomp_InterfaceA.c},
  which \emph{only} contains a \javacode{<clinit>} method if exist. 
  \end{enumerate}

\item During the code generation for an RJava class, 
the following information will be collected:
  \begin{enumerate}
  \item types and methods that are referenced. This will instruct the generation of 
  an \ccode{include} list.
  \item type header and implemented interface initialization code. 
  This consists the \ccode{rjava_class_init()} in
  addition to Java's \javacode{<clinit>}.
  \end{enumerate}

\item Name mangling in \rjc follows the following rules (see \javacode{CIdentifiers}):
  \begin{itemize}
  \item \textbf{package name and class name}: dot is replaced by underscore. 
  \\E.g. \javacode{org.yourcomp.A} $\rightarrow$ \ccode{org_yourcomp_A}.
  
  \item \textbf{method name}: \javacode{<init>} $\rightarrow$ \ccode{rjinit}, 
  \javacode{<clinit>} $\rightarrow$ \ccode{rjclinit}, others remain the same. 

  \item \textbf{method signature in definition}: package name, class name, method name, 
  parameter types, concatenated with underscore. 
  \\E.g. \javacode{void org.yourcomp.A.foo(int)} $\rightarrow$ 
  \ccode{void org_yourcomp_A_foo_int32_t(int)}.

  \item \textbf{function pointer in type header}: method name and parameter types,
  concatenated with underscore. 
  \\E.g. \javacode{void org.yourcomp.A.foo(int)} $\rightarrow$ 
  \ccode{void (*foo_int32_t)(int)}.
  
  \item \textbf{static field}: `var', class name, field name, concatenated with underscore. 
  (adding `var' to avoid clash with function names)
  \\E.g. \javacode{int a} in \javacode{org.yourcomp.A} $\rightarrow$ 
  \ccode{int var_org_yourcomp_A_a}.
  
  \item \textbf{nested class}: the same as Java name mangling. 
  However, dollar symbol is replaced by `\textunderscore NESTED \textunderscore ' when it appears in file name. 
  \\E.g. \javacode{org.yourcomp.A.NestedClass} $\rightarrow$ \ccode{org_yourcomp_A\$NestedClass}
  with file name \ccode{org_yourcomp_A_NESTED_NestedClass}.
  
  \end{itemize}
  
\item Method parameters are named as \ccode{parameterN} where N is the index of this parameter. 
An instance method has its first parameter as a pointer (\ccode{void*}) to its receiver, named as \ccode{this_parameter}. 

\item Main method uses C signature of `\ccode{int main(int argc, char** parameter0)}', 
where \ccode{parameter0} is a C array and is transformed into an RJava C array at the beginning
of main method by using a runtime helper \ccode{rjava_c_array_to_rjava_array()}. Before main method
returns, all threads will be joined (\ccode{rjava_join_all_threads()}). 

\end{itemize}

\subsubsection{Runtime}

Part of \rjc 's C runtime is hand coded library implementation written in C,
and the other part is generated during code generation, i.e. \ccode{rjava_crt.h}. 
\ccode{rjava_crt.h} is included by every C file generated by \rjc. 

\ccode{rjava_crt.h} includes:

\begin{itemize}
\item includes of C libraries, such as standard libraries (\ccode{stdio.h}, \ccode{stdlib.h}, \ccode{stdbool.h},
\ccode{inttypes.h}, \ccode{pthread.h} and \ccode{limits.h}) and external C libraries (hans-boehm GC). 

\item defines, such as OS define (\ccode{#define __OS_MACOSX_} or \ccode{#define __OS_LINUX_},
type define (\ccode{#define byte char}). 

\item \rjc object model, such as \ccode{RJava_Common_Class}, \ccode{RJava_Common_Instance},
and \ccode{RJava_Interface_Node}. 

\item type declarations of all application types. 

\item runtime global variables. 

\item runtime helper functions:
  \begin{itemize}
  \item \ccode{void rjava_class_init();}
  \item \ccode{void rjava_assert(bool cond, char* str);}
  \item \ccode{void rjava_add_interface_to_class(void* interface, int interface_size, char* name, RJava_Common_Class* klass);}
  \item \ccode{void rjava_alter_interface(void* interface, char* name, RJava_Common_Class* klass);}
  \item \ccode{void* rjava_get_interface(RJava_Interface_Node* list, char* name);}
  \item \ccode{void rjava_init_header(void* this_class, void* super_class, int super_class_size);}
  \item \ccode{void rjava_debug_print_header(void* this_class, char* name);}
  \item \ccode{void* rjava_new_array(int length, long ele_size);}
  \item \ccode{inline int rjava_length_of_array(void* array);}
  \item \ccode{inline void* rjava_access_array(void* array, int index);}
  \item \ccode{inline void* rjava_access_array_nobounds_check(void* array, int index);}
  \item \ccode{void* rjava_init_args(int argc, char** args);}
  \item \ccode{void* rjava_new_multiarray(int* dimensions, int dimension_size, long ele_size);}
  \item \ccode{void* rjava_c_array_to_rjava_array(int length, long ele_size, void* c_array);}
  \item \ccode{bool rjava_instanceof(void* instance, void* class_struct);}
  \item \ccode{void rjava_init_thread_suspending();}
  \item \ccode{void rjava_unimplemented_method();}
  \item \ccode{void rjava_runtime_global_init();}
  \end{itemize}

\item definitions of inlining helper funtions. 

\end{itemize}

\ccode{rjava_crt.c} contains the definitions of the other helper functions. 

\clearpage

%%%%%%%%%%%%%%%%%%
% MMTk user manual    %
%%%%%%%%%%%%%%%%%%
% 5. MMTk
\section{MMTk/RJava Manual}

MMTk updates with JikesRVM. However, current \rjc code base contains an unofficial 
branch of MMTk. Though MMTk has nothing to do with \rjc implementation,
MMTk is the most important project that is easily compatible with \rjc. 

\subsection{Unofficial Changes}

As described earlier, RJava originates from MMTk but is generalized and formalized. 
This makes current design of RJava is slightly different that in MMTk. 
We apply current design to MMTk code, and make a few changes to make it
fully compatible with \rjc, which is the unofficial branch of MMTk in \rjc code base. 

Having an unoffical branch is a temporary solution, and we seek to push those
changes back to JikesRVM's MMTk. However, this may happen after \rjc becomes 
more robust. 

The changes include:

\begin{itemize}
\item 
\textbf{@\javacode{MMTK} restriction ruleset\\}
Every classes in MMTk is annotated with this ruleset. 
\begin{lstlisting}[language=java]
@RestrictionRuleset

@RJavaCore
@NoRuntimeAllocation
@Uninterruptible
public @interface MMTk {
}
\end{lstlisting}

\item 
\textbf{Removing uses of \javacode{enum}\\}
RJava disallows the use of enumerate type. 
Thus \javacode{org.mmtk.utility.statistics.Stats.Phase} and
\javacode{org.mmtk.vm.ReferenceProcessor.Semantics} are removed,
and their use are replaced with an \javacode{int}. 

\item
\textbf{Removing use of \javacode{java.lang.Class}\\}
To eliminate any potential dynamic loading and reflection, 
RJava disallows any use of \javacode{java.lang.Class}. 
Though it is known that MMTk does not dynamically load
classes nor use reflection, MMTk has a few occurrences
of \javacode{java.lang.Class}. Changs are lised below:
  \begin{itemize}
  \item \javacode{CollectorContext} of each \javacode{Plan}. 
  For a \javacode{Plan}, its \javacode{CollectorContext} used to be
  calculated from class name, then retrieved by \javacode{Class.forName()}
  and new instances are created from from it. Since
  those uses are forbidden in RJava, the code is refactorred. 
  The changes involve \javacode{org.mmtk.plan.Plan.<clinit>},\\
  \javacode{org.mmtk.plan.ParallelCollectorGroup.initGroup()},
  \\and \javacode{org.mmtk.plan.Plan.newCollectorContext()} (newly
  added abstract method). 
  
  \item specialized scan class. 
  \javacode{org.mmtk.plan.TransitiveClosure} stores specialized scan
  class for current plan, however, MMTk does not directly use those
  classes but pass them to the hosting VM. Now MMTk simply stores
  class names of the specialized scan class, and let the hosting VM
  do the \javacode{Class.forName()} work. Changes involve
  \javacode{org.mmtk.plan.TransitiveClosure} and \\
  \javacode{org.mmtk.plan.Plan.getSpeicalizedScanClass()}. 
  
  \item MMTk-VM interface factory. 
  MMTk provides a set of abstract classes as MMTk-VM interface that 
  the hosting VM has to implement. \javacode{org.mmtk.vm.Factory}
  is responsible to create instances of the actual implementation (this
  happens in \javacode{org.mmtk.vm.VM}). 
  However, the factory class is instantiated from its class name, thus
  is forbidden. This is resolved by making 
  \javacode{org.mmtk.vm.VM} as a host-specific class so 
  that the factory is hard coded by the host VM. 
  \end{itemize}

\item
\textbf{Removing GCSpy and GCTrace\\}
The use of GCSpy and GCTrace in MMTk is obsolete, and is planned to
remove from its code base. For simpliciy, 
they are removed from \rjc 's branch. 

\end{itemize}

\subsection{VM-MMTk Interface}

MMTk and its hosting VM needs to interface with each other. 
The MMTk-VM interface is well specified, as the hosting VM
simply needs to implement the abstract classes in
\javacode{org.mmtk.vm} package. However, the VM-MMTk
interface seems subtle. 

This part describes how the hosting VM should
use MMTk as its memory manager. The description here
may be one of the many possible ways to use MMTk. 

\begin{itemize}
\item
\textbf{Plan selection\\}
MMTk provides different GC and allocation algorithms, which
are called `Plans'. The hosting VM needs to choose
a plan to use. To use \javacode{org.mmtk.plan.x.X}, 
the hosting VM should make \javacode{final} classes 
inherited from the plan \javacode{org.mmtk.plan.x.X}, the mutator \javacode{org.mmtk.plan.x.XMutator},
the collector \javacode{org.mmtk.plan.x.XCollector}, and the plan
constraints \javacode{org.mmtk.plan.x.XConstraints}. And the
hosting VM should use the selection classes instead of directly using
MMTk classes to keep hosting VM code independent of MMTk. 

\item
\textbf{Per-thread mutator context\\}
MMTk assume every thread in the hosting VM that wants to
allocate by using MMTk should contain an instance of 
mutator context, e.g. 
\javacode{MutatorSelect} (this is the \javacode{final} class
inherited from \javacode{org.mmtk.plan.x.XMutator}). To
achieve this, hosting VM can inherit its own thread class
from \javacode{MutatorSelect}, or store a reference to
\javacode{MutatorSelect} in each thread. Each thread
should initialize its mutator context by 
calling \javacode{mutator.initMutator(id)} with a unique 
mutator ID. 

\item
\textbf{Collector thread\\}
MMTk requires the hosting VM to spawn collector thread
on its behalf. Collector thread should contain not only
a mutator context(as stated above, mutator is supposed
to be `per-thread'), but also an instance of 
collector context, e.g. \javacode{CollectorSelect}. \javacode{collector.initCollector(id)}
can be used to initialize the collector instance. 

\item
\textbf{Booting sequence\\}
MMTk should be initialized before any allocation is required. 
A classical MMTk booting code is showed below:
\begin{lstlisting}[language=java]
// creating a global plan instance
plan = new PlanSelect();

// setting heap size
HeapGrowthManager.boot(INITIAL_HEAP_SIZE, MAX_HEAP_SIZE); 

// setting options
Options.x.setValue(x); 
...

// process options
PlanSelect.getPlan().processOptions();

// enabling allocation and collection
PlanSelect.getPlan().enableAllocation();
PlanSelect.getPlan().enableCollection();
\end{lstlisting}

\item
\textbf{Required GC header\\}
The hosting VM can freely choose its own object model,
and inform MMTk of the object model by implementing 
MMTk-VM interface. MMTk also requires a minimal GC
header for different chosen plans. The hosting
VM can get the required header size by 
\javacode{ConstraintsSelect.getConstraints().gcHeaderWords()},
which returns how many words MMTk requires as its
GC header. 

\item
\textbf{Allocation sequence\\}
Here shows a classical MMTk allocation sequence:
\begin{lstlisting}[language=java]
public static Address alloc(HostVMObject object) {
  // prepare arguments
  int allocator = PlanSelect.getPlan().ALLOC_DEFAULT; 
  int site = PlanSelect.getPlan().getAllocationSite(false); 
  int align = 0;
  int offset = 0;
  int size = HostVMObjectModel.getSize(object);

  // allocation
  MutatorSelect mutator = Scheduler.getCurrentThread().mutator(); 
  Address ret = mutator.alloc(size, align, offset, allocator, site);
  HostVMObjectModel.initialize(ret, object);
  mutator.postAlloc(ret.toObjectReference(), HostVMObjectModel.getType(object), size, allocator);
  
  return ret;
}
\end{lstlisting}

\item
\textbf{Object scanning\\}
During object reclamation, MMTk may require
the hosting VM to help with object scanning 
(the hosting VM needs to implement 
\javacode{org.mmtk.vm.Scanning}). 
In the implementation, the hosting VM can call the following
MMTk methods to proceed object scanning:
$1)$ \javacode{TransitiveClosure.processEdge()}, and 
$2)$ \javacode{TraceLocal.processRootEdge()}. 

\end{itemize}

The above discussed the basic VM-MMTk interface, which
should be suffice to implement a simple runtime with MMTk. 
The MMTk `testbed' is implemented by merely using the above
interface. More details and can be found in the code base. 

\vspace{2em}
\noindent
(End of the manual)
\clearpage

\end{document}
